\textbf{Gene expression metrics related to information content and robustness}
\textbf{(A)} Expression differences between cell types are compiled into a signal ratio matrix (SR) and binarized into a differentiation matrix (DM) reflecting whether each pair of cell types is distinguished (1) or not (0). The Differentiation Index (DI) is the fraction of nonzero values. The Signal Contrast (SC) is the average expression difference between distinguished pairs divided by the average expression difference between undistinguished pairs. 
\textbf{(B)} Highly significant ANOVA genes (warm colored dots) include a mixture of genes with high SC and low DI and genes with low SC and high DI. 
\textbf{(C)} Definition of generalized PSI (percent spliced in). For a splice donor, a generalized form of PSI (donor branch probability) can be defined as the joint distribution of transition probabilities from the donor to each acceptor. Acceptor branch probability can be defined conversely. 
\textbf{(D)} PANTHER \citep{Thomas_2003} gene families enriched in the top 1000 DI and the top 1000 SC genes. Red lines indicate the $p=10^{-5}$ threshold used to judge significance.
\textbf{(E)} Histogram of all donor branch probabilities from alternatively spliced sites. The distribution is highly bimodal, indicating that alternative splicing is "all or none" for each site in each cell type (though  often varying between cell types). 
\textbf{(F)} PANTHER gene families enriched in the top 500 DN genes. The number of cell types distinguished by a gene's splice variants (sDN; see Methods for calculation) rather than the ratio (DI) is used since the denominator of DI (total number of cell types potentially distinguished) varies for each gene. This is because genes not expressed in a cell type can contribute to distinctions based on expression, but not to those based on splicing.  Red lines indicate the $p=10^{-3}$ threshold used to judge significance.

