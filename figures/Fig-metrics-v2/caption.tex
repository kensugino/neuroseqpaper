\textbf{Gene expression metrics related to information content and robustness}
\textbf{(A)} Expression differences between cell types are compiled into a signal ratio matrix (SR) and binarized into a differentiation matrix (DM) reflecting whether each pair of cell types is distinguished (1) or not (0). The Differentiation Index (DI) is the fraction of nonzero values. The Signal Contrast (SC) is the average expression difference between distinguished pairs divided by the average expression difference between undistinguished pairs. 
\textbf{(B)} (Upper) An example of simulated binary and graded expression patterns with added noise. X-axis indicates sample/groups. (Lower) Various average metrics calculated from the simulated expression patterns (100 individual simulations; error bars are standard deviations). Values are normalized within each metric across binary expression group or graded expression group. 
\textbf{(C)} Summary of each metrics' correlation with Mutual Information and SNR: check mark--correlated, X--uncorrelated, triangle--partially correlated.
\textbf{(D)} Highly significant ANOVA genes (warm colored dots) include a mixture of genes with high SC and low DI and genes with low SC and high DI. 
\textbf{(E)} PANTHER \citep{Thomas_2003} gene families enriched in the top 1000 DI and the top 1000 SC genes. Red lines indicate the $p=10^{-5}$ threshold used to judge significance.
