\textbf{Features of DI for expression and splicing}

\textbf{(A) DI and MI are highly correlated.} The relationship between DI, calculated without considering replicates%(q-value part was dropped)\%
, and MI with expression levels discretized into 2 levels (left) and 5 levels (right). Although increasing the number of discrete expression levels decreases the degree of correlation, they remain monotonically and closely related.
\textbf{(B)} Definition of generalized PSI (percent spliced in). For a splice donor, a generalized form of PSI (donor branch probability) can be defined as the joint distribution of transition probabilities from the donor to each acceptor. Acceptor branch probability can be defined conversely. 
\textbf{(C)} Histogram of all donor branch probabilities from alternatively spliced sites. The distribution is highly bimodal, indicating that alternative splicing is "all or none" for each site in each cell type (though  often varying between cell types). 
\textbf{(D)} The number of cell types distinguished by a gene's splice variants (sDN; see methods for calculation). The number, rather than the ratio (DI) is calculated since the denominator of DI (total number of cell types potentially distinguished) varies for each gene. This is because genes not expressed in a cell type can contribute to distinctions based on expression, but not to those based on splicing.%For each branch of alternative splice sites, and a pair of cell types, we call they are "different" when 1) either all the replicates from one group are less than 0.3 and all the replicates from the other group are greater than 0.7, or vice versa and 2) both cell types in the pair have reasonable reads at the alternative site ($>10$ reads). Condition 1) is justified by the bimodal distribution shown in (C). Accumulating all pairs, these create a DM for each branch. We then combined branches to create a new DM for each gene (if any branch distinguishes a pair that pair is called "different" at the gene level). This DM represents how pairs of cell types are distinguished by any branches belonging to the gene. Since number of pairs actually compared may be different depending on gene (since some genes in some cell types may not be expressed), instead of DI which assumes fixed number of total pairs, we use DN (total number of pairs distinguished) to rank genes.\% 
PANTHER gene families enriched in the top 500 DN genes are shown. Red lines indicate the $p=10^{-3}$ threshold used to judge significance.
