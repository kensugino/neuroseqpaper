\textbf{Relationship between DI and MI/Splice DI}

\textbf{(A) DI and MI are highly correlated.} Relationship between DI, calculated without considering replicates (see Methods), and MI with expression levels discretized into 2 levels (left) and 5 levels (right). Although increasing the number of discrete expression levels decreases the degree of correlation, they remain monotonically and closely related.
\textbf{(B)} Definition of generalized PSI (percent spliced in). For a splice donor, a generalized form of PSI (donor branch probability) can be defined as a probability distribution with each component indicating transition probability from the donor to an acceptor. Acceptor branch probability can be similarly defined. 
\textbf{(C)} Histogram of all the donor branch probability from alternative splice sites (i.e. degree n > 1 ). The distribution is highly bimodal, indicating in a cell type, alternative splicing choice is unique for each site. 
\textbf{(D)} For each branch of alternative splice sites, and a pair of cell types, we call they are "different" when 1) either all the replicates from one group are less than 0.3 and all the replicates from the other group are greater than 0.7, or vice versa and 2) both cell types in the pair have reasonable reads at the alternative site (>10 reads). Condition 1) is justified by the bimodal distribution shown in (C). Accumulating all pairs, these create a DM for each branch. We then combined branches to create a new DM for each gene (if any branch distinguishes a pair that pair is called "different" at the gene level). This DM represents how pairs of cell types are distinguished by any branches belonging to the gene. Since number of pairs actually compared may be different depending on gene (since some genes in some cell types may not be expressed), instead of DI which assumes fixed number of total pairs, we use DN (total number of pairs distinguished) to rank genes. PANTHER gene families enriched in the top 500 DN genes are shown in (D). Red lines indicate the $p=10&{-3}$ threshold used to judge significance.
