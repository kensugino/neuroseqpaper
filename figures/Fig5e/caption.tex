\textbf{Mechanisms contributing to low noise and high information content of Homeobox TFs.} \textbf{(A)} Example expression patterns of a LIM class homeobox TF (Lhx1) and a calcium binding protein (Calb2) with similar overall expression levels. Sample key as in Figure 1 Supplements 1-3. \textbf{(B)} (upper) OFF state noise (defined as standard deviation (std) of samples with FPKM<1) plotted against maximum expression. (lower) HUGO gene groups enriched in the region indicated by red dashed lines in the upper panel (see Figure 5 Supplement 1 for PANTHER and Gene Ontology enrichments). \textbf{(C)} Average (replicate n=2) ATAC-seq profiles for the genes shown in A. Some peaks are truncated. Expression levels are plotted at right (grey bars). \textbf{(D)} Length-normalized ATAC profile for genes with high ($>$ 0.3, blue dashed box in B, n=853) and low ($<$ 0.2, red dashed box in B, n=1643) off state expression noise. \textbf{(E)} Each circle represents the separability of cell populations calculated using each PANTHER family. Separability is a measure of the expression distance between cell populations (defined as the average of 1- Pearson’s corr. coef.) calculated over a set of genes. Since dispersion of separability decreases with family size, results are compared to separability calculated from randomly sampled groups of genes (green solid lines: mean and std. dev.; green dashed lines: 99\% confidence interval). Z-scores: homeobox TF: 17.4, GPCR: 16.1, receptor: 13.1 and signaling molecule: 11.2.