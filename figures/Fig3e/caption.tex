\textbf{Gene expression metrics related to information content and robustness} \textbf{(A)} Cartoon illustrating the process of calculating Signal Contrast (SC) and Differentiation index (DI) for four different hypothetical genes that differ in their information content (2\&4 vs. 1\&3) and signal-to-noise ratio (SNR; 1\&2 vs. 3\&4) of their expression patterns across cell types. Expression signals are used to construct matrices for each gene of the signal differences between cell types (Signal ratio matrix) and the distinctions between cell types based on those differences (Differentiation Matrix; DM; see methods). The Differentiation Index (DI) is the fraction of the total pairs of cell types distinguished (i.e. of nonzero values in DM excluding diagonal). The Signal Contrast (SC) is the average expression difference between distinguished pairs divided by the average expression difference between undistinguished pairs. Orange and blue bars show that the resulting DI and SC calculations capture the variations in information and SNR across the four genes. \textbf{(B)} Highly significant ANOVA genes (warm colored dots) include a mixture of genes with high SC and low DI and genes with low SC and high DI. \textbf{(C),(D)} PANTHER gene families enriched in the top 1000 SC and top 1000 DI genes. Red lines indicate the p = $10^{-5}$ threshold used to judge significance.