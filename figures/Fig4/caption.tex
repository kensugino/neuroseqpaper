\textbf{Long genes have a greater capacity for differential expression. (A)} Black dots: DI of each gene is plotted against sorted gene length. Red dots: binned average of DIs (1000 genes per bin, sorted by length). \textbf{(B)} Average metrics for long ($\gt$100kbp) and short ($\lt$100kbp) neuronal genes (reproducibly expressed in neuronal cell types). \textbf{(C)} Separability of cell types calculated as in Figure 3E, but using long neuronal genes and short neuronal genes rather than functionally defined gene families. Z-score is 33.2 for long and 22.1 for short neuronal genes. Both are highly different from randomly sampled genes (green solid lines mean and Std. dev.; dashed lines = 99\% confidence interval), but long genes provide greater separation. \textbf{(D)} ATAC-seq peaks for Gabra5 showing different patterns of peaks for each of 7 cell types. Scale (top right) in reads per million. Expression levels for each cell type are shown at right (gray bars). \textbf{(E)} Black dots: number of distinct peak patterns observed across 7 ATAC-seq pro1led cell types plotted against the gene length for each gene; 7 corresponds to a distinct pattern for each pro1led cell type. Red dots: binned averages of black dots as in panel A. Background histograms show numbers of genes in each length bin. \textbf{(F)} Violin plot showing the relationship between DI and the number of different patterns of ATAC-seq peaks. Corr.coef. (0.31) is greater than that between DI and gene length (0.19; panel A).  