\textbf{(A)} Number (histogram) and ratios (dots) of genes expressed in neurons (pink histogram, red dots) and non-neurons (brown histogram, green dots) relative to the number of genes in the entire population (grey histogram) as a function of gene length (ratios computed per bin of 500 genes). 
\textbf{(B)} Number (cyan histogram) and ratios (cyan dots) of genes with nearby NRSE relative to the number of neuronally expressed genes (pink histogram). 
\textbf{(C)} (Magenta dots) ratio of neuronally expressed non-REST target genes to the population. Other components are same as in A.
\textbf{(D)} DI dependence of length without REST target genes compared to all genes.
\textbf{(E)} Fraction of gene length attributable to intron length.
\textbf{(F)} Length dependence of peak counts in the ATAC-seq data from the current study.
\textbf{(G)- (J)} Length dependence of peak counts in the ENCODE data. Examples from mouse ENCODE data in forebrain \textbf{(G)} and liver \textbf{(H)} tissue samples showing individual peaks (black dots) and binned averages (red dots) as a function of gene length. Average mouse \textbf{(I)} and human \textbf{(J)} peak counts from brain(blue) and non-brain(green) samples.
\textbf{(K)} Number of alternative site for each gene (in Gencode mouse v14) plotted against gene length.
\textbf{(L)} Similar to Figure 4 Supplement 1D, mean Pearson's correlation coefficients between genes within long and short gene groups.

%\textbf{(E)} PANTHER gene families enriched among long and short neuronal genes. 
%\textbf{(F)} Genomic regions occupied by long and short genes. Although long genes are fewer in number they occupy significantly more of the genome. Long \& short: regions in which long and short genes overlap. 



