\textbf{Properties of long genes in current and prior datasets.}\textbf{(A)} Number (histogram) and ratios (dots) of genes expressed in neurons (pink histogram, red dots) and non-neurons (brown histogram, green dots) relative to the number of genes in the entire population (grey histogram) as a function of gene length (ratios computed per bin of 500 genes). 
\textbf{(B)} Number (cyan histogram; left axis) and ratios (cyan dots; right axis) of genes with nearby NRSE relative to the numbers of neuronally expressed genes (pink histogram). 
\textbf{(C)} (Magenta dots) ratio of neuronally expressed non-REST target genes to the population. Other components are same as in A.
\textbf{(D)} DI dependence of length without REST target genes compared to all genes. DI is still strongly length dependent because REST targets are a small fraction of expressed long genes.
\textbf{(E)} Fraction of gene length attributable to intron length.
\textbf{(F)} Length dependence of peak counts in the ATAC-seq data from the current study.
\textbf{(G)- (J)} Length dependence of peak counts in ENCODE DNase hypersensitivity data. Examples from mouse ENCODE data in forebrain (telencephalon) \textbf{(G)} and liver \textbf{(H)} samples showing individual peaks (black dots) and binned averages (red dots) as a function of gene length. Average mouse \textbf{(I)} and human \textbf{(J)} peak counts from brain(blue) and non-brain(green) samples.
\textbf{(K)} Number of alternative splice sites for each gene (in Gencode mouse v14) plotted against gene length.
\textbf{(L)} Distribution of gene lengths for low OFF noise genes (Figure 4B red dashed region) and high OFF noise genes (Figure 4B blue dashed region). Red lines are medians and whiskers indicate 1.5 IQR. (***:p$<$1e-100, Student's t-test.)
\textbf{(M)} Similar to Figure 4 Supplement 1D, mean Pearson's correlation coefficients between genes within long and short gene groups relative to mean and S.D. (green solid lines) and 99\% confidence interval (green dashed lines) calculated from randomly selected groups of genes.

%\textbf{(E)} PANTHER gene families enriched among long and short neuronal genes. 
%\textbf{(F)} Genomic regions occupied by long and short genes. Although long genes are fewer in number they occupy significantly more of the genome. Long \& short: regions in which long and short genes overlap. 



