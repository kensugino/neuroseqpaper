\textbf{Gene expression metrics related to information content and robustness} \textbf{(A)} Cartoon illustrating the process of calculating fold-change ratio (FCR) and differentially expressed fraction (DEF) for four different hypothetical genes that differ in the information content (2\&4 vs. 1\&3) and signal-to-noise ratio (SNR; 1\&2 vs. 3\&4) of their expression patterns across cell populations. Expression signals are used to construct matrices for each gene of the log fold-changes between populations (fold-change matrix) and the distinctions between populations based on those differences (Differentiation Matrix; DM; see methods). The differentially expressed fraction (DEF) is the fraction of the total pairs of cell populations distinguished (i.e. of nonzero values in DM excluding diagonal). The fold-change ratio (FCR) is the average expression difference between distinguished pairs divided by the average expression difference between undistinguished pairs. Orange and blue bars show that the resulting DEF and FCR calculations capture the variations in information and SNR across the four genes.