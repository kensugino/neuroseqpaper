\textbf{Mechanisms contributing to high information content and low noise of Homeobox TFs.}
\textbf{(A)} Example expression patterns of a Lim class homeobox TF (LHX1) and a calcium binding protein (Calb2) with a similar overall expression level. 
(See method for details.)
\textbf{(B)} (upper) OFF state noise (defined as std. dev. of samples with RPKM$<$1) plotted against maximum expression. (lower) PANTHER families enriched in the region indicated by red dashed lines in the upper panel.
\textbf{(C)} Average (n=2) ATAC-seq profiles for the genes shown in A. Some peaks are truncated. Expression levels are plotted at right (grey bars)
\textbf{(D)} Length-normalized ATAC profile for genes with high (> 0.3, blue dashed box in B, n=853) and low (< 0.2, red dashed box in B, n=1643) off state expression noise.
\textbf{(E)} Mean separability for PANTHER families. Separability is as defined in Figure 2C. Metric used is 1-Pearson's correlation coefficient. Only families with number of members > 100 are shown, however, the rest had maximum separability of 4.9 (GABA receptor, #members=42) and did not affect the rank of homeobox TFs. 

