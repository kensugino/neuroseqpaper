\textbf{Mechanisms contributing to high information content and low noise of Homeobox TFs.}
\textbf{(A)} Example expression patterns of a Lim class homeobox TF (LHX1) and a calcium binding protein (Calb2) with a similar overall expression level. 
(See method for details.)
\textbf{(B)} (upper) OFF state noise (defined as std. dev. of samples with FPKM$<$1) plotted against maximum expression. (lower) PANTHER families enriched in the region indicated by red dashed lines in the upper panel.
\textbf{(C)} Average (n=2) ATAC-seq profiles for the genes shown in A. Some peaks are truncated. Expression levels are plotted at right (grey bars)
\textbf{(D)} Length-normalized ATAC profile for genes with high ($>$ 0.3, blue dashed box in B, n=853) and low ($<$ 0.2, red dashed box in B, n=1643) off state expression noise.
\textbf{(E)} Mean separability of cell types for PANTHER families. Separability between cell types for a PANTHER family is defined as 1- Pearson's corr.coef. using genes in the family. Green solid lines indicate mean and std. of separabilities calculated from randomly sampled genes and green dashed lines indicate 99\% confidence interval. Random samplings are from genes with at least one cell type with mean expression > 1 FPKM (N=30316). Z-score for homeobox TF is 17.4, GPCR is 16.1, receptor is 13.1 and signaling molecule is 11.2. 
