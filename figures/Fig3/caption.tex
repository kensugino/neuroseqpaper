\textbf{Mechanisms contributing to low noise and high information content of Homeobox TFs.} \textbf{(A)} Example expression patterns of a LIM class homeobox TF (Lhx1) and a calcium binding protein (Calb2) with similar overall expression levels. Cell type legend is given in Figure 1 Supplement 1. \textbf{(B)} (upper) OFF state noise (defined as std. dev. of samples with FPKM<1) plotted against maximum expression. (lower) PANTHER families enriched in the region indicated by red dashed lines in the upper panel. \textbf{(C)} Average (replicate N=2) ATAC-seq profiles for the genes shown in A. Some peaks are truncated. Expression levels are plotted at right (grey bars). \textbf{(D)} Length-normalized ATAC profile for genes with high (\textbf{$\gt$}\textbf{\textbf{\textbf{\textbf{}}}} 0.3, blue dashed box in B, n=853) and low ($\lt$ 0.2, red dashed box in B, n=1643) off state expression noise. \textbf{(E)} Mean separability of cell types for PANTHER families. Separability is a measure of gene expression distance (defined as the average of 1- Pearson’s corr. coef.) calculated across a set of genes. Since dispersion of separability decreases with family size, results are compared to separability calculated from randomly sampled groups of genes (green solid lines: mean and std. dev.; green dashed lines: 99\% confidence interval). Z-scores: homeobox TF: 17.4, GPCR: 16.1, receptor: 13.1 and signaling molecule: 11.2.