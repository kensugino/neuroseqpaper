\textbf{Genes are elongated by TE insertions and TEs contain information on gene expression levels}
\textbf{(A)} Histograms of segments inserted into human genome compared to chimp (left) and mouse genome compared to rat (right) by length (See methods). Peaks near 200bp corresponds to Alu, and near 6000bp corresponds to LINE. Pie charts (insets) indicate fraction of inserted bp overlapping transposable elements (TE) and other repeat elements.
\textbf{(B)} Percentage of ATAC peaks overlapping repeats. Left side: for all ATAC peaks, right side: for ATAC peaks overlapping recently inserted segments calculated in (A). 
\textbf{(C)} A schema for repeat score and regression model. Repeat score for a repeat and for an interval is just a number of the repeat residing in the interval (upper panel). The collection of repeat scores forms a scorematrix (X, lower panel) and used to fit gene expression levels (rank is used here). The regression calculates the intercept(a) and coefficients (b) which optimizes mean squared error or equivalently $R^2$. 
\textbf{(D)} The linear model is fitted using Ridge regression. The scorematrix for the fittings is calculated from genebody$+/-$10kb interval. Fitting is done for each cell type using 80\% of the genes and $R^2$ is calculated for the remaining 20\% which are not used for fitting. Shown in blue is the histogram of these test scores. For controls, fitting with several types of shuffling are calculated. Shown in green corresponds to the shuffling of rows in scorematrix (gene identities are shuffled ). Shown in orange corresponds to shuffling of gene locations keeping the gene length (so that the repeat score for the shuffled gene is changed but the relative location of repeats are kept). Shown in red corresponds to shuffling of repeat location to uniformly redistribute repeats. Last two essentially captures the correlation between gene length and expression (See Supp.Fig.XX)
\textbf{(E)} A model of 1) why neuronal genes are elongated over evolutionary time scales (tolerance to insertion), and 2) why TE insertions are an efficient way to create systems level diversity. 
