\textbf{Gene elongation across species is correlated with nervous system complexity}.
\textbf{(A)} Difference gene length distribution across species: human (Homo sapiens), chimp (Pan troglodytes), monkey (Macaca mulatta), mouse (Mus musculus), dog (Canis lupus familiaris), chicken (Gallus gallus), frog (Xenopus tropicalis), zebrafish (Danio rerio), fly (Drosophila melanogaster), and worm (Caenorhabditis  elegans) for all protein coding genes in each species (blue) and for the subset with unique homologs (yellow; see Methods). 
\textbf{(B)} Potassium channel Kcnma1 homolog length differences across species.
\textbf{(C)} Histograms of segments inserted into human genome compared to chimp (left) and mouse genome compared to rat (right) by length (See methods). Peaks near 200bp corresponds to Alu, and near 6000bp corresponds to LINE. Piecharts (insets) indicate fraction of inserted bp overlapping transposons and other repeat elements.
\textbf{(D)} A model of how neuronal gene elongation over evolutionary time scales enhances neuronal diversity. 
