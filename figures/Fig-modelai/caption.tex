%%width=0.75\columnwidth
\textbf{Genes are elongated by TE insertions and TEs contain information for gene expression}
\textbf{(A)} Distribution of gene length for various well annotated species. Red lines indicate means and whiskers indicate inter-quartile range. Blue bars are all protein coding genes and yellow bars are the subset of genes with homologs in all species. (human: \textit{Homo sapiens}; chimp: \textit{Pan troglodytes}; monkey: \textit{Macaca mulatta}, mouse: \textit{Mus musculus}; dog: \textit{Canis lupus familiaris}; chicken: \textit{Gallus gallus}; frog: \textit{Xenopus tropicalis}; zebrafish: \textit{Danio rerio}; fly: \textit{Drosophila melanogaster}; worm: \textit{Caenorhabditis  elegans})
\textbf{(B)} Histograms of lengths of segments inserted into the human genome compared to chimp (left) and mouse genome compared to rat (right). Peak near 300bp (more visible in human) corresponds to Alu, and near 6000bp corresponds to LINE. Pie charts (insets) indicate fraction of inserted bp overlapping transposable elements (TE) and other types of repeats. Gorilla and Guinea pig are used as surrogates of common ancestors of human and chimp, and mouse and rat, respectively (see Methods). 
\textbf{(C)} Percentage of ATAC peaks overlapping major categories of repeat elements. Left side: all ATAC peaks, right side: ATAC peaks overlapping recently inserted segments calculated in (B). 
\textbf{(D)} Schema describing repeat score and regression model. Repeat scores (upper panel) are calculated separately for each type of repeat element and for each gene as the count of that element in the specified interval determined by the gene. Regressions (lower panel) are calculated separately for each cell type by fitting coefficients (b) to ranked expression levels (Y) using intercept(a) and repeat score (X). 
\textbf{(E)} Fits to 80\% of the genes are cross validated using the remaining 20\%. Histograms show cross validated $R^2$ for each cell type (blue), and for controls shuffling the relationship between repeat scores and genes(score matrix; green) or changing the repeat score by randomly changing the location of repeats (red) or by calculating the repeat score over a randomly selected genomic interval of the same length as the gene (orange). The latter two shuffling methods retain some predictive value compared to shuffling the repeat score matrix (green) since they maintain the correlation between gene length and expression (See Figure 7 Supplement 1C).
\textbf{(F)} A model of how neuronal genes become elongated over evolutionary time scales. 
