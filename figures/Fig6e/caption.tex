\textbf{Long genes have a greater capacity for distinguishing cell populations. (A)} DEF as a function of gene length. For violin plots in A, B, D, genes are sorted by length and binned (4 bins per log unit). \textbf{(B)} Robustness of expression difference (FCR) as a function of gene length. \textbf{(C)} Separability of cell types calculated as in Figure 5E, but using long neuronal genes (> 1e5, n=1829, $\geq$100kb) and short neuronal genes (<1e5, n=10572, $\leq$100kb) rather than functionally defined gene families. Z-score is 33.2 for long and 22.1 for short neuronal genes. Both are highly different from randomly sampled genes (green solid lines mean and Std. dev.; dashed lines = 99\% confidence interval), but long genes provide greater separation. \textbf{(D)} Splice DEF as a function of gene length. \textbf{(E)} Fraction of pairs distinguished by splicing (splice-only), transcript abundance (exp-only), or by both measures \textbf{(F)} Variation in long gene expression in neuronal and non-neuronal populations across major brain regions studied.  