%%width=0.8\columnwidth
\textbf{Long genes have a greater capacity for differential expression.}
\textbf{(A)} (Black dots) DI of each gene is plotted against sorted gene length. (Red dots) binned average ()1000 genes per bin).
\textbf{(B)} Fraction of longest genes expressed in neuronal (red bars) and non-neuronal cell types (blue bars) within each brain region profiled.
\textbf{(C)} Number of cell types with distinct patterns of peaks observed in 7 ATAC samples plotted against the gene length
\textbf{(D)} Violin plot showing the relationship between DI and the number of cell types having different patterns of peaks. 
\textbf{(E)} ATAC-seq peaks for Gabra5 showing different patterns of peaks for each cell type. Expression levels in these cell types are shown at right (gray bars)
\textbf{(F)} Bar graphs showing mean DI, bDI and SC for long and short neuronal genes (reproducibly expressed in neurons and gene length $\geq$100kbp or $<$100kbp respectively). 
\textbf{(G)} Similar to Figure 4E. Mean distance of pairs of cell types calculated using long neuronal genes and short neuronal genes. Z-score is 33.2 for long and 22.1 for short neuronal genes. 
