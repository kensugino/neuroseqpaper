\textbf{Long genes have a greater capacity for differential expression.}
\textbf{(A)} (Black dots) DI of each gene is plotted against sorted gene length. (Red dots) binned average with bin size of 1000 genes.
\textbf{(B)} DI bias replotted over an expanded scale for all neuronally expressed genes (Red dots) and following removal of REST target genes (black dots).
\textbf{(C)} Number of cell types with distinct patterns of peaks observed in 7 ATAC samples plotted against the gene length
\textbf{(D)} A violin plot showing the relationship between DI and the number of cell types having different patterns of peaks. 
\textbf{(E)} ATAC-seq peaks for Gabra5 showing different patterns of peaks for each cell type. Expression levels in these cell types are shown at right (gray bars)
\textbf{(F)} Heatmaps showing the separability of each pair of cell types calculated using long neuronal genes (expressed in neurons and gene length $geq$ 100kbp; left) and short neuronal genes (gene length $<$ 100kbp). Order of the cell types are the same as Figure 1 Supplement 1 and also indicated by color coding at the bottom. 
\textbf{(G)} (Left) Mean separability of pairs of cell types calculated using long neuronal genes and short neuronal genes. Error bars are standard deviations. (Right) Percentage of pairs with separability $>$ 2. 
