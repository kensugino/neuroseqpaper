\textbf{Long genes have a greater capacity for differential expression.}
\textbf{(A)} (Black dots) DI of each gene is plotted against sorted gene length. (Red dots) binned average with bin size of 1000 genes.
\textbf{(B)} DI bias replotted over an expanded scale for all neuronally expressed genes (Red dots) and following removal of REST target genes (black dots).
\textbf{(C)} Number of cell types with distinct patterns of peaks observed in 7 ATAC samples plotted against the gene length
\textbf{(D)} A violin plot showing the relationship between DI and the number of cell types having different patterns of peaks. 
\textbf{(E)} ATAC-seq peaks for Gabra5 showing different patterns of peaks for each cell type. Expression levels in these cell types are shown at right (gray bars)
\textbf{(F)} Bar graphs showing mean DI, bDI and SC for long and short neuronal genes (reproducibly expressed in neurons and gene length $geq$ 100kbp or $<$ respectively). 
\textbf{(G)} Similar to Figure 4E. Mean distance of pairs of cell types calculated using long neuronal genes and short neuronal genes. Z-socre is 33.2 for long and 22.1 for short neuronal genes. 
