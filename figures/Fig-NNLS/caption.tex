\textbf{Decomposition by NNLS.}
\textbf{(A)} NNLS coefficients of NeuroSeq cell types by two SCRS datasets.
\textbf{(B)} (Left) Tasic et al. clusters decomposed by Zeisel et al. clusters. (Right) Zeisel et al. clusters decomposed by Tasic et al. clusters. There are few perfect matches.
%\textbf{(C)} Definition of separability. Separability of a pair of groups is the ratio of distance between the centroids and the sum of diameters of the groups. Distance between two points is calculated as 1-correlation coefficients of the two points. The diameter of a group is calculated as the sum of the mean distance from the centroid and the 3 times standard deviation of the distances from the centroids. 
%\textbf{(D)} Heatmaps showing the separability of each pair of cell types defined by clusters (SCRS datasets) or by replicates (NeuroSeq). The order of the groups in each matrix are the same as in panel (A). The upper row color range is such that blue regions indicate pairs that are either touching or overlapping. The same data are plotted using an expanded color scale in the lower row to illustrate the dynamic range of separability in the three studies.
\textbf{(C)} Mean purity scores for NeuroSeq and SCRS datasets. The purity score for a sample is defined as the ratio of the highest coefficient to the sum of all coefficients. (**:p$<$0.01, t-test.)



