\textbf{Gene expression metrics related to information content and robustness} \textbf{(A)} Cartoon illustrating the process of calculating Signal Contrast (SC) and Differentiation index (DI) for four different hypothetical genes that differ in the information content (right vs. left quadrants) and signal-to-noise ratio (SNR; upper vs. lower quadrants) of their expression patterns across cell types. Expression signals (log TPM) are used to construct matrices of the signal differences between cell types (Signal difference matrix) and the distinctions between cell types based on those differences (Differentiation Matrix; DM; see methods). The Differentiation Index (DI) is the fraction of the total pairs of cell types distinguished (i.e. of nonzero values in DM). The Signal Contrast (SC) is the average expression difference between distinguished pairs divided by the average expression difference between undistinguished pairs. \textbf{(B)} Highly significant ANOVA genes (warm colored dots) include a mixture of genes with high SC and low DI and genes with low SC and high DI. \textbf{(C),(D)} PANTHER gene families enriched in the top 1000 SC and top 1000 DI genes. Red lines indicate the p = 10*5 threshold used to judge significance. 