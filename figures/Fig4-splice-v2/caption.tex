\textbf{Alternative splicing and neuronal diversity. (A)} (Top) Schematic representation of branch probabilities. Alternative donor sites (red dot) can be spliced to multiple acceptor sites $1,\ldots, m$ with probabilities $p_{1},\ldots,p_{m}$. (Bottom) Distribution of branch probabilities across all samples and genes. \textbf{(B)} Heatmap showing branch probabilities across neuronal samples for branches with highest Splice DEF. Each row corresponds to a branch within the indicated gene on the left and the location is indicated on the right. Samples without junctional reads at this branch are colored white. \textbf{(C)} Enriched HUGO gene groups and PANTHER protein classes for genes with top 1000 combined splice DEF. \textbf{(D)} Splice graphs illustrating examples of alternative splicing leading to inclusion or exclusion (marked "i","e") of Pfam domains (magenta exons) with branch probabilities shown in the heatmap below. Previously unannotated exons and junctions are blue; annotated are black. Dotted lines indicate branches predicted to lead to nonsense-mediated decay (NMD). A red star above an exon indicates existence of a premature termination codon (PTC) within the exon which satisfies the "50nt rule" for NMD \citep{Nagy_1998}, whereas a black star indicates existence of a PTC within 50bp of the next junction. Dashed lines and hatches indicate that there is no coding path through the element. (>) indicates an annotated translation start site. \textbf{(E)} Proportion of branch points predicted to lead to NMD (purple), altered Pfam inclusion (red), or both (overlapped region), at one or more of its branches.
