%%width=0.8\columnwidth
\textbf{Supplementary to Figure 7. Repeats elongate genes and contain information on gene expression}
\textbf{(A)} Example of gene length difference across species for Kcnma1 (slo, potassium ion channel). 
\textbf{(B)} Evolutionary age of human genes correlates with their length. The length distribution of human genes is plotted as a function of age, estimated from their most distant homologs. Genes common to all vertebrates (or to all listed genomes) are longer than genes common only to mammals or common only to primates.
\textbf{(C)} Correlation between gene expression rank and gene length (blue) and SINE repeat score (orange) calculated for all cell types. 
\textbf{(D)} Similar to Fig.7D but using repeat scores calculated from different sized intervals surrounding each gene. Average $R^2$ is maximal near 10kb for both upstream and downstream. Shuffling conditions are colored as in Fig. 7D.
\textbf{(E)} Similar to Fig.7D but for repeat scores calculated from genebody only (upper panel) or genebody$+/-$100kbp (lower panel).
\textbf{(F)} Introns account for most of the length of long genes. 
\textbf{(G)} Fraction of genome devoted to long genes (orange) is greater than that devoted to short genes (green), despite being fewer in number. Some genomic regions contain overlapping long and short genes (yellow).
\textbf{(H)} Percentage of inserted sequences calculated in Fig.7A (Human vs. Chimp and Mouse vs. Rat), that overlap TEs within long ($\geq$ 100kbp) or short ($<$100kbp) genes. 