\textbf{Genomic properties transposable elements (TEs).} 
\textbf{(A)} Introns account for most of the length of long genes. 
\textbf{(B)} Fraction of genome devoted to long genes (orange) is greater than that devoted to short genes (green), despite being fewer in number. Some genomic regions contain overlapping long and short genes (yellow).
\textbf{(C)}Percentage of ATAC peaks that overlaps with RepeatMasker TE separated by the age of the segments containing each peak. The age of the segment is inferred from the multiz/net alignments downloaded from the UCSC Genome Browser for the indicated genomes: Mm10 (Mus musculus), Rn6 (Rattus norvegicus), CavPor3 (Cavia porcellus, Guinea pig), OryCun2 (Oryctolagus cuniculus, Rabbit), Hg38 (Homo Sapiens), MonDom5 (Monodelphis domestica, Opossum), GalGal5 (Gallus gallus, chicken) and DanRer7 (Danio rerio, Zebrafish). 
\textbf{(D)} For inserted sequences calculated using trios used in Figure 7C (Human-Chimp-Gorilla and Mouse-Rat-Guinea pig), percentage of inserted sequences that overlap with TE which are inserted into long ($\geq$ 100kbp) or short ($<$100kbp) genes are shown. 