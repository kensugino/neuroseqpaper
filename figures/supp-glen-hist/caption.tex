\textbf{Genomic properties transposable elements (TEs).} 
\textbf{(A)} Introns account for most of the length of long genes. 
\textbf{(B)} Fraction of genome devoted to long genes (orange) is greater than that devoted to short genes (green), despite being fewer in number. Some genomic regions contain overlapping long and short genes (yellow).
\textbf{(C)} Percentage of ATAC peaks  overlapping with RepeatMasker TE as a function of the estimated age of the segments containing each peak. Ages are inferred from the multiz/net alignments downloaded from the UCSC Genome Browser for the indicated genomes: Mm10 (Mus musculus), Rn6 (Rattus norvegicus), CavPor3 (Cavia porcellus, Guinea pig), OryCun2 (Oryctolagus cuniculus, Rabbit), Hg38 (Homo Sapiens), MonDom5 (Monodelphis domestica, Opossum), GalGal5 (Gallus gallus, chicken) and DanRer7 (Danio rerio, Zebrafish). 
\textbf{(D)} Percentage of inserted sequences calculated in Figure 7C (Human vs. Chimp and Mouse vs. Rat), that overlap TEs within long ($\geq$ 100kbp) or short ($<$100kbp) genes. 