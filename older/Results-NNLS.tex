\subsection{Comparison to single cell datasets}
Pools of sorted neurons may be heterogeneous if multiple neuronal subtypes are labeled in the same brain region of the same strain. SCRS has recently emerged as a viable method for profiling cellular diversity that does not suffer from this limitation. However, since profiles of cell types in SCRS studies are obtained by clustering individual, often noisy, cellular profiles, inaccuracies can arise from misclustering or overclustering. In order to assess the relative cellular homogeneity of our sorted samples, we compared the current dataset to the cluster profiles from SCRS studies. We focused on neuronal and non-neuronal cell types in the neocortex, profiled in two recent studies \citep{Tasic_2016,Zeisel_2015}. Assuming  each sorted population corresponds to a linear combination of one or more SCRS profiles, we assessed homogeneity by linear decomposition using non-negative least squares (NNLS). We performed multiple checks on the validity of the procedure (see Figure 2 Supplements 1-3 and Methods) and found that it is able to fairly accurately decompose mixtures of component expression profiles when those components are well separated. 

For each sorted cell type, the procedure identifies the weights (coefficients) of  component clusters (cell types) from the SCRS datasets (Figure 2A). As expected, cell types present in the SCRS studies, but not profiled in NeuroSeq, (e.g. L4 neurons, VIP interneurons and oligodendrocytes), were not matched (purely blue columns in Figure 2A). Other cell types matched perfectly to a single SCRS cell type (e.g., microglia, astrocytes, ependyma) or matched to more than one, implying heterogeneity in the sorted profiles or poor separation of the SCRS profiles. Profiles with imperfect matches usually matched closely related cell types. For example, the NeuroSeq Pvalb interneuron group matched one or two of the SCRS Pvalb-positive interneuron clusters, and layer 2/3 (L2/3) pyramidal neurons matched SCRS L2/3 clusters, or an adjacent cluster in L4 (Tasic: L4 Arf5). The spread of coefficients repeatedly involved the same few SCRS cell clusters (e.g. columns L5b Tph2 and L5b Cdh13 in Tasic; and S1PyrL5,  S1PyrL6 in Zeisel), which could occur if these clusters are not well separated, which we confirmed by a cross-validation procedure (Figure 2 Supplement 3). We measured the "purity" of the decomposition as the fractional match to the highest coefficient. The purity scores for the decomposition of NeuroSeq cell types by the two SCRS datasets were higher than those obtained for SCRS cell clusters decomposed by the other SCRS data set (Figure 2B,C). This implies that although sorted sample heterogeneity may exist in some of our sorted samples, it is comparable (or smaller) than the inaccuracies introduced by clustering single cell profiles. We also compared the separability of cell types assayed in the sorted and SCRS datasets (Figure 2 Supplement 4) by calculating the gene expression distances between each cell type within each dataset. NeuroSeq profiles were far more separable than clusters in either SCRS dataset, likely because of the noise reduction achieved by averaging across cells and because of the larger numbers of cells and reads comprising each profile. Hence sorted and single cell techniques have complimentary strengths and cross referencing both data modalities may provide the most accurate assessment of cell type specific expression. 



