\subsection{Long genes contribute disproportionately to neuronal diversity}

We found that neuronal effector genes (ion channels, receptors and cell adhesion molecules) have the greatest ability to distinguish cell types (Figure 2D). Previously, these categories of genes have been found to be selectively enriched in neurons and to share the physical characteristic of being long \citep{Sugino_2014,Gabel_2015,Zylka_2015}. Consistent with this, DI is strongly correlated with length (Figure 4A). Analyzing publicly available single cell data confirms that this bias is broadly observable (Figure 4 Supplement 1). 

Long genes ($>$100kb) have nearly twice the average ability to distinguish cell types (DI) as shorter genes (Figure 4B), and provide greater separation between cell types, despite being nearly four times fewer in number (Figure 4C). In contrast, the robustness of expression differences between cell types (SC) is not different for long and short genes (Figure 4B). 

The enhanced contribution of long genes to neuronal diversity indicates that these genes are differentially regulated across a large number of cell types. Gene regulation in the brain depends on interactions between promoters and distal enhancers that may be located intergenically or within introns \citep{Kim_2010,Gray_2017}. We hypothesized that since long genes contain larger numbers of longer introns (Figure 4 supplement 2A), they may harbor a larger number of intronic regulatory elements. Such regulatory elements are frequently sites of enhanced chromatin accessibility \cite{Harrow_2012}. 
%Previous work showing that long genes are preferentially expressed in the brain is perplexing given the metabolic costs associated with long genes \cite{Castillo_Davis_2002}. Our finding that long genes contribute disproportionately to distinguishing neuronal cell types suggests that this may be a counterbalancing benefit. Long genes are long because they contain a larger number as well as longer introns (Figure 4 supplement 2A). We hypothesized that increased length could contribute to neuronal diversity through diversity in gene regulation as the increased non-coding sequence within these genes could provide a platform for the evolution of novel regulatory elements.
To determine differences between candidate regulatory elements across cell types, we identified sites of enhanced genome accessibility using ATAC-seq \cite{Buenrostro_2013} from 7 different neuronal cell types. As expected, long genes had more candidate regulatory elements (ATAC peaks; Figure 4 supplement 2B) and these peaks were present in a greater number of distinct patterns per gene across cell types (Figure 4D-E). Consistent with a role in differential expression, the number of unique patterns correlated with the degree of differential expression across cell types (Figure 4F). Thus, long genes contribute disproportionately to neuronal diversity, likely by diversifying  gene regulation via an increased number of regulatory sites within their introns. 


