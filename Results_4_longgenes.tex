\subsection{Long genes contribute disproportionately to neuronal diversity}

We found that neuronal effector genes (ion channels, receptors and cell adhesion molecules, etc.) have the greatest ability to distinguish cell populations (Figure 4E). Previously, these categories of genes have been found to be selectively enriched in neurons and to share the physical characteristic of being long \citep{Sugino_2014,Gabel_2015,Zylka_2015}. Consistent with this, DEF is significantly correlated with length (Figure 7A; p=7.5e-189). Analyzing publicly available single cell data confirms that this bias is broadly observable (Figure 7 Supplement 1). In contrast, the robustness of expression between cell types (FCR) is higher in shorter genes (e.g. between 3-30kb) albeit much less significantly compared to DEF (Figure 7B). Long genes ($\geq$100kb) have nearly twice the average ability to distinguish cell populations (DEF) as shorter genes, and provide greater family-wise separation between cell types (Figure 7C).  

In addition to being differentially expressed, long genes are likely to have a greater potential for differential splicing. To evaluate the degree to which differential splicing of long genes contributes to distinguishing cell populations we plotted the splice DEF (Figure 6) as a function of gene length. As expected, DEF calculated from differential splicing also increased with gene length (Figure 7D) although the slope was more gradual and the maximum DEF value achieved was less than that for gene expression (Figure 7A). For each gene, we measured the fraction of cell populations pairs that could be distinguished on the basis of differential expression, differential splicing, or both. This revealed that for the current dataset, the average alternatively spliced gene distinguishes only 1.4 \% of cell populations, but distinctions based on expression of these same genes were much more common (13.9 \%, Figure 7E). 

Finally, to determine whether neuronal long gene expression contributes more to profiles in some anatomical regions than in others, we plotted the fraction of longest genes expressed in neuronal and nonneuronal populations across each of the major brain regions studied. The results confirm strong differences between neurons and nonneurons and show the strongest long gene expression in forebrain regions, with weaker expression evident in hindbrain (Figure 7F). Analyses of single cell datasets revealed similar trends (Figure 7 Supplement 2). 


%The enhanced contribution of long genes to neuronal diversity indicates that these genes are differentially regulated across a large number of cell types. One mode of gene regulation depends on interactions between promoters and distal enhancers that may be located intergenically or within introns \citep{Kim_2010,Gray_2017}. We hypothesized that since long genes contain larger numbers of longer introns (Figure 5 supplement 2A), they may harbor a larger number of intronic regulatory elements. Such regulatory elements are frequently sites of enhanced chromatin accessibility \citep{Harrow_2012}. 
%Previous work showing that long genes are preferentially expressed in the brain is perplexing given the metabolic costs associated with long genes \cite{Castillo_Davis_2002}. Our finding that long genes contribute disproportionately to distinguishing neuronal cell types suggests that this may be a counterbalancing benefit. Long genes are long because they contain a larger number as well as longer introns (Figure 4 supplement 2A). We hypothesized that increased length could contribute to neuronal diversity through diversity in gene regulation as the increased non-coding sequence within these genes could provide a platform for the evolution of novel regulatory elements.
%To determine differences between candidate regulatory elements across cell types, we identified sites of enhanced genome accessibility using ATAC-seq \citep{Buenrostro_2013} from 7 different neuronal cell types. As expected, long genes had more candidate regulatory elements (ATAC peaks; Figure 5 supplement 2B, also Figure 5 Supplement 2C-F) and these peaks were present in a greater number of distinct patterns per gene across cell types (Figure 5D-E). Consistent with a role in differential expression, the number of unique patterns correlated with the degree of differential expression across cell types (Figure 5F). Thus, long genes contribute disproportionately to neuronal diversity, likely by diversifying  gene regulation via an increased number of regulatory sites within their introns. 


