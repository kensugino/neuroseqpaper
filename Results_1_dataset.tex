\section{Results}

\subsection{A dataset of genetically-identified neuronal transcriptomes}
To identify genes contributing most to mammalian neuronal diversity, we collected transcriptomes from 179 genetically and anatomically identified populations of neurons and 15 populations of non-neuronal cells in mice (Table 1; Figure 1; Figure 1 Supplement 1; Supplementary Table 1,2). The great majority (186/194) were identified both genetically and anatomically, with the remaining identified only anatomically, by their location and projection patterns. Each collected population represents a group of fluorescently labeled cells dissociated and sorted from a specific micro-dissected region of the mouse brain or other tissue. The pipeline for cell type-specific transcriptome collection is depicted in Figure 1A (see Methods for additional details). Mouse lines were first characterized by generating a high-resolution atlas of reporter expression (Figure 1B) then, regions containing labeled cells with uniform morphology were chosen for sorting and RNA-seq. In total, we sequenced 2.3 trillion bp in 565 libraries. This effort constitutes the largest and most diverse single collection of genetically identified cell populations profiled by RNA-seq. The raw data is deposited to NCBI GEO (GSE79238). The processed data, including anatomical atlases, RNA-seq coverage, and TPM are available at http://neuroseq.janelia.org (Figure 1C).

To determine the sensitivity of our transcriptional profiling, we used ERCC spike-ins. Amplified RNA libraries had an average sensitivity (50\% detection) of 23 copy*kbp of ERCC spike-ins across all libraries (Figure 1D). Since manually sorted samples had 132$\pm$16 cells (mean$\pm$ sem, all following as well), this indicates our pipeline had the sensitivity to detect a single copy of a transcript per cell 80\% of the time. This high sensitivity allowed for deep transcriptional profiling in our diverse set of cell populations.

To assess the extent of contamination in the dataset, we checked expression levels of marker genes for several non-neuronal cell types (Figure 1 Supplement 2B). As previously shown \citep{Okaty_2011}, manual sorting produced, in general, extremely clean data.

To assess the homogeneity of the sorted, pooled samples, we compared our datasets to publicly available single cell (SC) datasets. To compare across different datasets, we used a method based on linear decomposition by non-negative least squares (NNLS) (See Figure 2 and Figure 2 Supplements 1-6). This method tests the degree to which individual profiles can be decomposed into linear mixtures of profiles from another dataset. Such mixture or impurities can arise in at least two ways (Figure 2A): by pooling similar cell types prior to sequencing in the case of sorted datasets, or by pooling similar profiles after sequencing, at the clustering stage, in the case of SC datasets. Although NNLS is a widely used decomposition procedure, it has not previously been applied to expression profiles and so we did a number of control experiments to validate its use. First, we cross-validated the decompositions by dividing each dataset in half and testing the ability to decompose one half by the other (Figure 2 supplement 1). This revealed that some neuroSeq samples had overlapping coefficients and so could not be well distinguished. For example, pairs of populations identified in layer 2/3 of two different regions in the same strain (AI.L23\_glu\_P157 / ORBm.L23\_glu\_P157) or by retrogradely labeling cells in the same layer and region from two different targets (SSp.L23\_glu\_M1.inj / SSp.L23\_glu\_S2.inj and SSp.L5\_glu\_BPn.inj / SSp.L5\_glu\_IRT.in) are hard to distinguish. On the other hand, overlapping coefficients were also present for some pairs of cell populations in the SC datasets (such as Oligo Serpinb1a / Oligo Synpr in the Tasic dataset and MGL1 / MGL2 / MGL3 in the Zeisel dataset). On average the purity, defined as how well a single sample can be decomposed into the most closely corresponding sample, was similar across the three datasets (Figure 2 Supplement 1D). We also demonstrated that NNLS decomposition could be used to recover the numbers of cell types isolated from distinct strains in a SC dataset, after mixing these profiles together, despite the fact that this information was not included in the fitting procedure (Figure 2 Supplement 2). NNLS (Figure 2B,C) produced comparable or cleaner decompositions than a competeing Random Forest algorithm (Figure 2 Supplement 7). These results indicate that NNLS can be used to reliably decompose mixtures of cellular profiles. Similar average coefficients (i.e. similar purity) were obtained for decomposition of the neuroSeq data by SC datasets and by decomposing these datasets by each other (Figure 2, Figure 2 Supplements 3-6). Hence our decomposition results indicate that although heterogeneity may exist in some of our sorted samples, it is comparable to the inaccuracies introduced by clustering SC profiles.

Since merging or splitting of closely related clusters either prior to sequencing or during the clustering process can lead to poor discrimination between samples, we also measured the separability of cell population profiles obtained in each study (Figure 2 Supplement 8). As expected, the clusters of sorted samples, which are averages across one hundred cells or more, were much more cleanly separable than SC clusters. Taken together, NNLS decomposition and separability provide a quantitative framework for assessing the trade-offs between homogeneity and reproducibility when measuring population transcriptomes from GACPs and SCs. 

To demonstrate the utility of the dataset, made possible by its broad sampling of neuronal cell populations, we extracted pan-neuronal genes (genes expressed commonly in all neuronal cell populations but expressed at lower levels or not at all in non-neuronal cell populations; Figure 1 Supplement 3). Here, broad sampling of cell populations is essential to avoid false positives \citep{Zhang_2014,Mo_2015,Stefanakis_2015}. Because of the high sensitivity and low noise, we were able to be conservative and exclude genes expressed in most but not all neuron types. Extracted pan-neuronal genes contain well known genes such as \textit{Eno2} (Enolase2), which is the neuronal form of Enolase required for the Krebs cycle, \textit{Slc2a3} (chloride transporter) required for inhibitory transmission, and \textit{Atp1a3} (ATPase Na+/K+ transporting subunit alpha 3) which belongs to the complex responsible for maintaining electrochemical gradients across the membrane, as well as genes not previously known to be pan-neuronal, such as \textit{2900011O08Rik} (now called Migration Inhibitory Protein; \citealt{Zhang_2014a}). Synaptic genes are often differentially expressed among neurons, but interestingly some included in this pan-neuronal list such as \textit{Syn1, Stx1b, Stxbp1, Sv2a}, and \textit{Vamp2} appear to be common components, highlighting essential parts of these complexes. Thus, the dataset should be useful for many other applications, especially those requiring comparisons across a wide variety of neuronal cell types. 

\begin{table}[p]
%%\begin{fullwidth}
\caption{\label{tab:table1}Summary of Profiled Samples.}
% Use "S" column identifier to align on decimal point 
%\begin{tabular}{S l l l r}
%%for_pdf%%\resizebox{1\columnwidth}{!}{
\begin{tabular}{l l l l l l}%{ | l | l | l | l | l | l | }
\toprule
	 & region/type & transmitter & \#groups & subregions & \#samples \\ 
\midrule
	CNS neurons & Olfactory (OLF) & glu & 10 & AOBmi,MOBgl,PIR,AOB,COAp & 30 \\ 
	 &  & GABA & 4 & AOBgr,MOBgr,MOBmi & 11 \\ 
	 & Isocortex & glu & 22 & VISp,AI,MOp5,MO,VISp6a,SSp,SSs,ECT,ORBm,RSPv & 68 \\ 
	 &  & GABA & 3 & Isocortex,SSp (Sst+, Pvalb+) & 7 \\ 
	 &  & glu,GABA & 1 & RSPv & 3 \\ 
	 & Subplate (CTXsp) & glu & 1 & CLA & 4 \\ 
	 & Hippocampus (HPF) & glu & 24 & CA1,CA1sp,CA2,CA3,CA3sp,DG,DG-sg,SUBd-sp,IG & 65 \\ 
	 &  & GABA & 4 & CA3,CA,CA1 (Sst+, Pvalb+) & 12 \\ 
	 & Striatum (STR) & GABA & 12 & ACB,OT,CEAm,CEAl,islm,isl,CP & 33 \\ 
	 & Pallidum (PAL) & GABA & 1 & BST & 4 \\ 
	 & Thalamus (TH) & glu & 11 & PVT,CL,AMd,LGd,PCN,AV,VPM,AD & 29 \\ 
	 & Hypothalamus (HY) & glu & 11 & LHA,MM,PVHd,SO,DMHp,PVH,PVHp & 36 \\ 
	 &  & GABA & 4 & ARH,MPN,SCH & 15 \\ 
	 &  & glu,GABA & 2 & SFO & 3 \\ 
	 & Midbrain (MB) & DA & 2 & SNc,VTA & 5 \\ 
	 &  & glu & 2 & SCm,IC & 6 \\ 
	 &  & 5HT & 2 & DR & 10 \\ 
	 &  & GABA & 1 & PAG & 4 \\ 
	 &  & glu,DA & 1 & VTA & 3 \\ 
	 & Pons (P) & glu & 7 & PBl,PG & 22 \\ 
	 &  & NE & 1 & LC & 2 \\ 
	 &  & 5HT & 2 & CSm & 7 \\ 
	 & Medulla (MY) & GABA & 7 & AP,NTS,MV,NTSge,DCO & 18 \\ 
	 &  & glu & 6 & NTSm,IO,ECU,LRNm & 20 \\ 
	 &  & ACh & 2 & DMX,VII & 6 \\ 
	 &  & 5HT & 1 & RPA & 3 \\ 
	 &  & GABA,5HT & 1 & RPA & 4 \\ 
	 &  & glu,GABA & 1 & PRP & 3 \\ 
	 & Cerebellum (CB) & GABA & 10 & CUL4, 5mo,CUL4, 5pu,CUL4, 5gr,PYRpu & 25 \\ 
	 &  & glu & 4 & CUL4, 5gr,NODgr & 10 \\ 
	 & Retina & glu & 5 & ganglion cells (MTN,LGN,SC projecting) & 14 \\ 
	 & Spinal Cord & glu & 1 & Lumbar (L1-L5) dorsal part & 3 \\ 
	 &  & GABA & 4 & Lumbar (L1-L5) dorsal part, central part & 12 \\ 
	PNS & Jugular & glu & 2 & (TrpV1+) & 7 \\ 
	 & Dorsal root ganglion (DRG) & glu & 2 & (TrpV1+, Pvalb+) & 5 \\ 
	 & Olfactory sensory neurons (OE) & glu & 4 & MOE,VNO & 9 \\ 
\midrule     
	non-neuron & Microglia &  & 2 & MOp5(Isocortex),UVU(CB) (Cx3cr1+) & 6 \\ 
	 & Astrocytes &  & 1 & Isocortex (GFAP+) & 4 \\ 
	 & Ependyma &  & 1 & Choroid Plexus & 2 \\ 
	 & Ependyma &  & 2 & Lateral ventricle (Rarres2+) & 6 \\ 
	 & Epithelial  &  & 1 & Blood vessel (Isocortex) (Apod+,Bgn+) & 3 \\ 
	 & Epithelial &  & 1 & olfactory epithelium & 2 \\ 
	 & Progenitor &  & 1 & DG (POMC+) & 3 \\ 
	 & Pituitary &  & 1 & (POMC+) & 3 \\ 
\midrule     
	non brain & Pancreas &  & 2 & Acinar cell, beta cell & 7 \\ 
	 & Myofiber &  & 2 & Extensor digitorum longus muscle & 7 \\ 
	 & Brown adipose cell&  & 1 & Brown adipose cell from neck.  & 4 \\ 
\midrule  
	 &  & total & 194 &  & 565 \\ 
\bottomrule
\end{tabular}
%%for_pdf%%}
%\end{fullwidth}
\end{table}




