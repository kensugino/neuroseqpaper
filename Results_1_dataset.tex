\section{Results}

\subsection{A dataset of genetically-identified neuronal transcriptomes}
To identify genes contributing most to mammalian neuronal cell type diversity, we collected transcriptomes from 179 types of neurons and 15 types of non-neuronal cell populations in mice (Table 1; Figure 1; Figure 1 Supplement 1; Supplementary Table 1,2). The great majority (186/194) were identified genetically and anatomically, with the remaining identified by their projection patterns. Each collected neuron type represents a group of fluorescently labeled cells dissociated and sorted from a specific micro-dissected region of the mouse brain or other tissue. The pipeline for cell type-specific transcriptome collection is depicted in Figure 1A (see Methods for additional details). Mouse lines were first characterized by generating a high-resolution atlas of reporter expression (Figure 1B) then, regions containing labeled cells with uniform morphology were chosen for sorting and RNA-seq. In total, we sequenced 2.3 trillion bp in 565 libraries. This effort constitutes the largest and most diverse single collection of genetically identified cell types profiled by RNA-seq. The raw data is deposited to NCBI GEO (GSE79238). The processed data, including anatomical atlases, RNA-seq coverage, and TPM are available at http://neuroseq.janelia.org (Figure 1C).

To determine the sensitivity of our transcriptional profiling, we used ERCC spike-ins. Amplified RNA libraries had an average sensitivity (50\% detection) of 23 copy*kbp of ERCC spike-ins across all libraries (Figure 1D). Since manually sorted samples had 132$\pm$16 cells (mean$\pm$ sem, all following as well), this indicates our pipeline had the sensitivity to detect a single copy of a transcript per cell 80\% of the time. This high sensitivity allowed for deep transcriptional profiling in our diverse set of cell types.

To assess the extent of contamination in the dataset, we checked expression levels of marker genes for several non-neuronal cell types (Figure 1 Supplement 2B). As previously shown \citep{Okaty_2011}, manual sorting produced, in general, extremely clean data.

To assess the homogeneity of the sorted, pooled samples, we compared our datasets to publicly available single cell datasets. To compare across different datasets, we used a method based on linear decomposition by non-negative least squares (NNLS) (See Figure 2 and Figure 2 Supplements 1-4). The results indicate that although sorted sample heterogeneity may exist in some of our sorted samples, it is comparable (or smaller) than the inaccuracies introduced by clustering single cell profiles.

To demonstrate the utility of the dataset, made possible by its broad sampling of cell types, we extracted pan-neuronal genes (genes expressed commonly in all neuronal cell types but expressed at lower levels or not at all in non-neuronal cell types; Figure 1 Supplement 3). Here, broad sampling is essential to avoid false positives \citep{Zhang_2014,Mo_2015,Stefanakis_2015}. Because of the high sensitivity and low noise (Figure 1 Supplement 5), we were able to be conservative and exclude genes expressed in most but not all neuron types. Extracted pan-neuronal genes contain well known genes such as \textit{Eno2} (Enolase2), which is the neuronal form of Enolase required for the Krebs cycle, \textit{Slc2a3} (chloride transporter) required for inhibitory transmission, and \textit{Atp1a3} (ATPase Na+/K+ transporting subunit alpha 3) which belongs to the complex responsible for maintaining electrochemical gradients across the membrane, as well as genes not previously known to be pan-neuronal, such as \textit{2900011O08Rik} (now called Migration Inhibitory Protein; \cite{Zhang_2014a}). Synaptic genes are often differentially expressed among neurons, but interestingly some included in this pan-neuronal list such as \textit{Syn1, Stx1b, Stxbp1, Sv2a}, and \textit{Vamp2} appear to be common components, highlighting essential parts of these complexes. Thus, the dataset should be useful for many other applications, especially those requiring comparisons across a wide variety of neuronal cell types. 

\begin{table}[p]
%%\begin{fullwidth}
\caption{\label{tab:table1}Summary of Profiled Samples.}
% Use "S" column identifier to align on decimal point 
%\begin{tabular}{S l l l r}
%%for_pdf%%\resizebox{1\columnwidth}{!}{
\begin{tabular}{l l l l l l}%{ | l | l | l | l | l | l | }
\toprule
	 & region/type & transmitter & \#groups & subregions & \#samples \\ 
\midrule
	CNS neurons & Olfactory (OLF) & glu & 10 & AOBmi,MOBgl,PIR,AOB,COAp & 30 \\ 
	 &  & GABA & 4 & AOBgr,MOBgr,MOBmi & 11 \\ 
	 & Isocortex & glu & 22 & VISp,AI,MOp5,MO,VISp6a,SSp,SSs,ECT,ORBm,RSPv & 68 \\ 
	 &  & GABA & 3 & Isocortex,SSp (Sst+, Pvalb+) & 7 \\ 
	 &  & glu,GABA & 1 & RSPv & 3 \\ 
	 & Subplate (CTXsp) & glu & 1 & CLA & 4 \\ 
	 & Hippocampus (HPF) & glu & 24 & CA1,CA1sp,CA2,CA3,CA3sp,DG,DG-sg,SUBd-sp,IG & 65 \\ 
	 &  & GABA & 4 & CA3,CA,CA1 (Sst+, Pvalb+) & 12 \\ 
	 & Striatum (STR) & GABA & 12 & ACB,OT,CEAm,CEAl,islm,isl,CP & 33 \\ 
	 & Pallidum (PAL) & GABA & 1 & BST & 4 \\ 
	 & Thalamus (TH) & glu & 11 & PVT,CL,AMd,LGd,PCN,AV,VPM,AD & 29 \\ 
	 & Hypothalamus (HY) & glu & 11 & LHA,MM,PVHd,SO,DMHp,PVH,PVHp & 36 \\ 
	 &  & GABA & 4 & ARH,MPN,SCH & 15 \\ 
	 &  & glu,GABA & 2 & SFO & 3 \\ 
	 & Midbrain (MB) & DA & 2 & SNc,VTA & 5 \\ 
	 &  & glu & 2 & SCm,IC & 6 \\ 
	 &  & 5HT & 2 & DR & 10 \\ 
	 &  & GABA & 1 & PAG & 4 \\ 
	 &  & glu,DA & 1 & VTA & 3 \\ 
	 & Pons (P) & glu & 7 & PBl,PG & 22 \\ 
	 &  & NE & 1 & LC & 2 \\ 
	 &  & 5HT & 2 & CSm & 7 \\ 
	 & Medulla (MY) & GABA & 7 & AP,NTS,MV,NTSge,DCO & 18 \\ 
	 &  & glu & 6 & NTSm,IO,ECU,LRNm & 20 \\ 
	 &  & ACh & 2 & DMX,VII & 6 \\ 
	 &  & 5HT & 1 & RPA & 3 \\ 
	 &  & GABA,5HT & 1 & RPA & 4 \\ 
	 &  & glu,GABA & 1 & PRP & 3 \\ 
	 & Cerebellum (CB) & GABA & 10 & CUL4, 5mo,CUL4, 5pu,CUL4, 5gr,PYRpu & 25 \\ 
	 &  & glu & 4 & CUL4, 5gr,NODgr & 10 \\ 
	 & Retina & glu & 5 & ganglion cells (MTN,LGN,SC projecting) & 14 \\ 
	 & Spinal Cord & glu & 1 & Lumbar (L1-L5) dorsal part & 3 \\ 
	 &  & GABA & 4 & Lumbar (L1-L5) dorsal part, central part & 12 \\ 
	PNS & Jugular & glu & 2 & (TrpV1+) & 7 \\ 
	 & Dorsal root ganglion (DRG) & glu & 2 & (TrpV1+, Pvalb+) & 5 \\ 
	 & Olfactory sensory neurons (OE) & glu & 4 & MOE,VNO & 9 \\ 
\midrule     
	non-neuron & Microglia &  & 2 & MOp5(Isocortex),UVU(CB) (Cx3cr1+) & 6 \\ 
	 & Astrocytes &  & 1 & Isocortex (GFAP+) & 4 \\ 
	 & Ependyma &  & 1 & Choroid Plexus & 2 \\ 
	 & Ependyma &  & 2 & Lateral ventricle (Rarres2+) & 6 \\ 
	 & Epithelial  &  & 1 & Blood vessel (Isocortex) (Apod+,Bgn+) & 3 \\ 
	 & Epithelial &  & 1 & olfactory epithelium & 2 \\ 
	 & Progenitor &  & 1 & DG (POMC+) & 3 \\ 
	 & Pituitary &  & 1 & (POMC+) & 3 \\ 
\midrule     
	non brain & Pancreas &  & 2 & Acinar cell, beta cell & 7 \\ 
	 & Myofiber &  & 2 & Extensor digitorum longus muscle & 7 \\ 
	 & Brown adipose cell&  & 1 & Brown adipose cell from neck.  & 4 \\ 
\midrule  
	 &  & total & 194 &  & 565 \\ 
\bottomrule
\end{tabular}
%%for_pdf%%}
%\end{fullwidth}
\end{table}




