\subsection{Information-based metrics to quantify differential expression}

Analysis of expression differences between individual groups is the basis of most profiling efforts. Variance-based metrics, such as Analysis of Variance (ANOVA) F-Value, or coefficient of variation (CV) are commonly used for this purpose. These metrics are jointly affected by the information content of the differential expression (pattern) and the robustness of the differences (effect size) and so cannot readily separate these two parameters. As a complement to these metrics, we developed two easily calculated metrics that better separate the information content and the robustness of expression differences. 

First, in order to extract the transcriptional signals related to cell type identity, we quantified each gene's ability to differentiate each pair of profiled cell types. Based on expression levels and variability (Figure 3A; Methods) we compiled a Differentiation Matrix (DM) with elements equal to one or zero depending on whether or not the gene is differentially expressed between each pair of profiles. The Differentiation Index (DI) is simply the fraction of pairs distinguished, excluding self-comparisons; and ranges from 0 to 1. The maximum observed value of 0.65 indicates that the gene distinguishes 65\% of the pairs, while a value of 0 indicates that the gene distinguishes none (i.e., expressed at similar levels in all cell types).

The ability to detect transcriptional differences between cell types depends on both magnitude of difference and associated noise. To quantify this in our second metric, we defined the Signal Contrast (SC), which closely reflects Signal-to-Noise-Ratio (SNR). Since the signals we are interested in are the gene expression differences distinguishing cell types, we used a noise estimate derived from all undistinguished pairs from the same gene. SC, which indicates how robustly pairs are distinguished, is the ratio of the average effect size for distinguished and undistinguished pairs. High SC genes robustly distinguish cell populations and are therefore suitable as "marker genes". 

Our metrics outperform existing metrics such as ANOVA, CV, and Fano factor in distinguishing the information content and robustness of differential expression. To illustrate the properties of DI and SC relative to existing metrics, we calculated these metrics against various simulated expression patterns with added noise (Figure 3B). The results (Figure 3B, lower) demonstrate that DI (blue) is highly correlated with mutual information (MI; green), yet much easier to calculate. This makes intuitive sense, since the division of cell types into those that can and cannot be distinguished (DM; Figure 3A) corresponds to a unit of information about cell types provided by a gene expression pattern (for more details of the relationship between DI and MI, see Figure 3 Supplement 1A and 2). 
%As shown below (Methods), in the simplified case of binary expression levels and ignoring noise, DI is an order-preserved transformation of MI, although MI is usually significantly more difficult to calculate precisely. More generally, DI is highly correlated with MI and can be regarded as an approximation to MI (For another metric related to DI, see Simpson's index of diversity \cite{SIMPSON_1949}). 
The simulations also show that DI is fairly independent from SNR. For example, both high and low SNR binary patterns yield same DIs. In contrast, SC (orange) is independent from MI, but is highly correlated to SNR. Thus, DI provides an estimate of the information content of expression patterns across cell types, whereas SC provides an estimate of SNR.

Unlike DI and SC, traditional variance-based methods like ANOVA F-values and CV are either affected by both MI and SNR (ANOVA) or by neither (CV). These differences between metrics are summarized in Figure 2C. The fact that ANOVA does not distinguish between information content and SNR is also apparent in the data. As shown in Figure 3D, high-ANOVA genes include both high DI and high SC genes. Therefore, SC and DI are useful because they provide independent measures of the robustness and magnitude of differential expression between cell types.






















