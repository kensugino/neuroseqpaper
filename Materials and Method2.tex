%\subsection{TF Tree}
%The set of mouse TFs was constructed by combining 4 curated TF lists: genes annotated in 1) PANTHER \citep{Thomas_2003} PC00218 (transcription factor), 2) Riken Transcription Factor Database \citep{Kanamori_2004}, 3) HUGO \citep{Gray_2014} families with TF functions and 4) Gene Ontology \citep{Ashburner_2000} GO:0006355 (regulation of transcription). Genes appearing reproducibly in these list (i.e. in more than 1 list) were used as TFs. Anatomical regions used as constraints are defined in a hierarchical manner (see Supplementary Table 5). 

%The TF tree is constructed recursively using the following algorithm:
%\begin{verbatim}
%preparation:
%    0. calculate bDIs for all subsets of samples defined by anatomical regions
%function bisect(list of samples):
%    1. if the list of samples consists of only one cell type, exit
%    2. calculate bDI,SC within this group of samples for all TFs
%    3. if there is no TF with bDI>0,  exit
%    4. find the appropriate level in the hierarchy of anatomical regions
%    5. penalize bDIs (from 2.) with bDIs of containing anatomical regions (from 0.)
%    6. sort TFs by their penalized bDI and SC in descending order
%    7. set candidates as TFs with penalized bDI>0.2, if there are none, take the top 5
%    8. for each candidate, calculate divisions of samples according to expression level
%       - at sample level, assign ON/OFF using FPKM=3 as threshold
%       - at cell type level, assign ON/OFF according to dominant ON/OFF of samples
%       - divide all cell types into ON or OFF groups
%       - optionally constrain division to anatomical boundary
%    9. if there is no division, exit
%    10. if there is more than one division then
%       - calculate "division strength" for all divisions:
%           - a0 = mean number of binary distinctions of all genes between ON and OFF groups
%           - a1 = mean number of binary distinctions of all genes within ON or OFF groups
%           - division strength = a0/a1
%       - then choose the division with the highest division strength
%    11. output ON/OFF groups and corresponding TF(s) for the chosen division
%    12. call bisect on ON group samples
%    13. call bisect on OFF group samples
%\end{verbatim}

\subsection{Inserted segments}
The multiz alignments downloaded from the UCSC genome browser \citep{Kent_2002} was used to calculate inserted segments in human or mouse. By comparing closely related species (human vs. chimp or mouse vs. rat), candidate segments inserted into human (or mouse) are extracted. By using another closely related species as a common ancestor (gorilla, guinea pig respectively for human/chimp and mouse/rat), segments absent in chimp and gorilla (or absent in rat/guinea pig) are called insertion in human (or mouse), and segments absent in chimp but present in gorilla (or absent in rat but present in guinea pig) are called deletion in chimp (or rat). 

\subsection{TE fitting}
Repeat annotations for mouse mm10 genome as detected by RepeatMasker \citep{repeatmasker} with Repbase \citep[ver. 20140131][]{Bao_2015} were used. Only repeat families with number of instances$>$200 are included. For individual repeats, only those with number of instances$>$50 are included. For repeats in the "Simple repeat" class, only those with number of instances$>$1000 are included. Repeat scores are calculated as described in Figure 7D using Gencode.vM13. Only genes with non-zero repeat scores are used for fitting. For fitting expression level (rank) by repeat score, a regularized version of linear regression, Ridge regression, was implemented in the Python scikit-learn library \citep{scikit-learn}.

\subsection{Tissue data}
In addition to cell type-specific data obtained in this study, we analyzed publicly available RNA-seq and DNase-seq data using tissue samples. Information on these samples are described in Supplementary Table 4.

\subsection{Annotations}
For reference annotations we used Gencode.vM13 \citep{Harrow_2012} downloaded from http://www.gencodegenes.org/, NCBI RefSeq \citep{Pruitt_2013} downloaded from the UCSC genome browser.

\subsection{Anatomical Region Abbreviations}
Region abbreviations: AOBmi, Accessory olfactory bulb, mitral layer; MOBgl, Main olfactory bulb, glomerular layer; PIR, Piriform area; COAp, Cortical amygdalar area, posterior part; AOBgr, Accessory olfactory bulb, granular layer; MOBgr, Main olfactory bulb, granular layer; MOBmi, Main olfactory bulb, mitral layer; VISp, Primary visual area; AI, Agranular insular area; MOp5, Primary motor area, layer5; VISp6a, Primary visual area, layer 6a; SSp, Primary somatosensory area; SSs, Supplemental somatosensory area; ECT, Ectorhinal area; ORBm, Orbital area, medial part; RSPv, Retrosplenial area, ventral part; ACB, Nucleus accumbens; OT, Olfactory tubercle; CEAm, Central amygdalar nucleus, medial part; CEAl, Central amygdalar nucleus, lateral part; islm, Major island of Calleja; isl, Islands of Calleja; CP, Caudoputamen; CA3, Hippocampus field CA3; DG, Hippocampus dentate gyrus; CA1, Hippocampus field CA1; CA1sp, Hippocampus field CA1, pyramidal layer; SUBd-sp, Subiculum, dorsal part, pyramidal layer; IG, Induseum griseum; CA, Hippocampus Ammon’s horn; PVT, Paraventricular nucleus of the thalamus; CL, Central lateral nucleus of the thalamus; AMd, Anteromedial nucleus, dorsal part; LGd, Dorsal part of the lateral geniculate complex; PCN, Paracentral nucleus; AV, Anteroventral nucleus of thalamus; VPM, Ventral posteromedial nucleus of the thalamus; AD, Anterodorsal nucleus; RT, Reticular nucleus of the thalamus; MM, Medial mammillary nucleus; PVH, Paraventricular hypothalamic nucleus; PVHp, Paraventricular hypothalamic nucleus, parvicellular division; SO, Supraoptic nucleus; DMHp, Dorsomedial nucleus of the hypothalamus, posterior part; ARH, Arcuate hypothalamic nucleus; PVHd, Paraventricular hypothalamic nucleus, descending division; SCH, Suprachiasmatic nucleus; LHA, Lateral hypothalamic area; SFO, Subfornical organ; VTA, Ventral tegmental area; SNc, Substantia nigra, compact part; SCm, Superior colliculus, motor related; IC, Ingerior colliculus; DR, Dorsal nucleus raphe; PAG, Periaqueductal gray; PBl, Parabrachial nucleus, lateral division; PG, Pontine gray; LC, Locus ceruleus; CSm, Superior central nucleus raphe, medial part; AP, Area postrema; NTS, Nucleus of the solitary tract; MV, Medial vestibular nucleus; NTSge, Nucleus of the solitary tract, gelatinous part; DCO, Dorsal cochlear nucleus; NTSm, Nucleus of the solitary tract, medial part; IO, Inferior olivary complex; VII, Facial motor nucleus; DMX, Dorsal motor nucleus of the vagus nerve; RPA, Nucleus raphe pallidus; PRP, Nucleus prepositus; CUL4,5mo, Cerebellum lobules IV-V, molecular layer; CUL4,5pu, Cerebellum lobules IV-V, Purkinje layer; PYRpu, Cerebellum Pyramus (VIII), Purkinje layer; CUL4,5gr, Cerebellum lobules IV-V, granular layer; MOE, main olfactory epithelium; VNO, vemoronasal organ.


