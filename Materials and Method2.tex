\subsection{TF Tree}
A set of TFs were constructed by combining 4 curated TF lists: genes annotated in 1) PANTHER \citep{Thomas_2003} PC00218 (transcription factor), 2) Riken Transcription Factor Database \citep{Kanamori_2004}, 3) HUGO \citep{Gray_2014} families with TF functions and 4) Gene Ontology \citep{Ashburner_2000} GO:0006355 (regulation of transcription). Genes appearing reproducibly in these list (i.e. in more than 1 list) were used as TF. 

The TF tree is constructed recursively using following algorithm:
input: 



\subsection{Inserted segments}


\subsection{TE fitting}


\subsection{Tissue data}
In addition to cell type-specific data obtained in this study, we analyzed publicly available RNASeq and DNase-seq data using tissue samples. Information on these samples are described in Supplementary Table 3.

\subsection{Annotations}
For reference annotations we used Gencode.vM13 (Harrow 2012) downloaded from http://www.gencodegenes.org/, NCBI RefSeq (Pruitt 2013) and UCSC known genes both downloaded from the UCSC genome browser.

\subsection{Anatomical Region Abbreviations}
Region abbreviations: AOBmi, Accessory olfactory bulb, mitral layer; MOBgl, Main olfactory bulb, glomerular layer; PIR, Piriform area; COAp, Cortical amygdalar area, posterior part; AOBgr, Accessory olfactory bulb, granular layer; MOBgr, Main olfactory bulb, granular layer; MOBmi, Main olfactory bulb, mitral layer; VISp, Primary visual area; AI, Agranular insular area; MOp5, Primary motor area, layer5; VISp6a, Primary visual area, layer 6a; SSp, Primary somatosensory area; SSs, Supplemental somatosensory area; ECT, Ectorhinal area; ORBm, Orbital area, medial part; RSPv, Retrosplenial area, ventral part; ACB, Nucleus accumbens; OT, Olfactory tubercle; CEAm, Central amygdalar nucleus, medial part; CEAl, Central amygdalar nucleus, lateral part; islm, Major island of Calleja; isl, Islands of Calleja; CP, Caudoputamen; CA3, Hippocampus field CA3; DG, Hippocampus dentate gyrus; CA1, Hippocampus field CA1; CA1sp, Hippocampus field CA1, pyramidal layer; SUBd-sp, Subiculum, dorsal part, pyramidal layer; IG, Induseum griseum; CA, Hippocampus Ammon’s horn; PVT, Paraventricular nucleus of the thalamus; CL, Central lateral nucleus of the thalamus; AMd, Anteromedial nucleus, dorsal part; LGd, Dorsal part of the lateral geniculate complex; PCN, Paracentral nucleus; AV, Anteroventral nucleus of thalamus; VPM, Ventral posteromedial nucleus of the thalamus; AD, Anterodorsal nucleus; RT, Reticular nucleus of the thalamus; MM, Medial mammillary nucleus; PVH, Paraventricular hypothalamic nucleus; PVHp, Paraventricular hypothalamic nucleus, parvicellular division; SO, Supraoptic nucleus; DMHp, Dorsomedial nucleus of the hypothalamus, posterior part; ARH, Arcuate hypothalamic nucleus; PVHd, Paraventricular hypothalamic nucleus, descending division; SCH, Suprachiasmatic nucleus; LHA, Lateral hypothalamic area; SFO, Subfornical organ; VTA, Ventral tegmental area; SNc, Substantia nigra, compact part; SCm, Superior colliculus, motor related; IC, Ingerior colliculus; DR, Dorsal nucleus raphe; PAG, Periaqueductal gray; PBl, Parabrachial nucleus, lateral division; PG, Pontine gray; LC, Locus ceruleus; CSm, Superior central nucleus raphe, medial part; AP, Area postrema; NTS, Nucleus of the solitary tract; MV, Medial vestibular nucleus; NTSge, Nucleus of the solitary tract, gelatinous part; DCO, Dorsal cochlear nucleus; NTSm, Nucleus of the solitary tract, medial part; IO, Inferior olivary complex; VII, Facial motor nucleus; DMX, Dorsal motor nucleus of the vagus nerve; RPA, Nucleus raphe pallidus; PRP, Nucleus prepositus; CUL4,5mo, Cerebellum lobules IV-V, molecular layer; CUL4,5pu, Cerebellum lobules IV-V, Purkinje layer; PYRpu, Cerebellum Pyramus (VIII), Purkinje layer; CUL4,5gr, Cerebellum lobules IV-V, granular layer; MOE, main olfactory epithelium; VNO, vemoronasal organ.


