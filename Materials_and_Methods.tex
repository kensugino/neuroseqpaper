\section{Materials and Methods}

\subsection{Cell types and mouse lines}
We assume that cell types are organized hierarchically in a tree-like fashion proceeding from major branches (e.g. "cortical excitatory neuron") to more specialized subtypes, with the terminal "leaf-level" branches comprising "atomic" cell types. Profiled cell types are defined operationally by the intersection of a transgenic mouse strain (or in some cases anatomical projection target) and a brain region. These "operational cell types" may or may not correspond to "atomic" cell types, but as shown in Figure 1 Supplement 3 have comparable purity to clusters of single cells. Mouse lines profiled in this study are summarized in Supplementary Table 1. Most were obtained from GENSAT \citep{Gong_2007} or from the Brandeis Enhancer Trap Collection \citep{Shima_2016}. For Cre-driver lines, the Ai3, Ai9 or Ai14 reporter \citep{Madisen_2009} was crossed and offspring hemizygous for Cre and the reporter gene were used for profiling. All experiments were conducted in accordance with the requirements of the Institutional Animal Care and Use Committees at Janelia Research Campus and Brandeis University.

\subsection{Tissue data}
In addition to cell type-specific data obtained in this study, we analyzed publicly available RNA-seq and DNase-seq data using tissue samples. Information on these samples are described in Supplementary Table 3.

\subsection{Atlas}
Animals were anesthetized and perfused with 4\% paraformaldehyde and brains were sectioned at 50$\mu m$ thickness. Every fourth section was mounted on slides and imaged with a slide scanner equipped with a 20x objective lens (3DHISTECH; Budapest, Hungary). In house programs were used to adjust contrast and remove shading caused by uneven lighting. Images were converted to a zoomify compatible format for web delivery and are available at http://neuroseq.janelia.org.

\subsection{Cell sorting}
Manual cell sorting was performed as described \citep{Hempel_2007, Sugino_2014}. Briefly, animals were sacrificed following isoflurane anesthesia, and 300$\mu m$ slices were digested with pronase E (1mg/ml, P5147; Sigma-Aldrich) for 1 hour at room temperature, in artificial cerebrospinal fluid (ACSF) containing 6,7-dinitroquinoxaline-2,3-dione (20$\mu M$; Sigma-Aldrich), D-(–)-2-amino-5-phosphonovaleric acid (50$\mu M$; Sigma-Aldrich), and tetrodotoxin (0.1$\mu M$; Alomone Labs). Desired brain regions were micro-dissected and triturated with Pasteur pipettes of decreasing tip size. Dissociated cell suspensions were diluted 5-20 fold with filtered ACSF containing fetal bovine serum (1\%; HyClone) and poured over Petri dishes coated with Sylgard (Dow Corning). For dim cells, Petri dishes with glass bottoms were used. Fluorescent cells were aspirated into a micropipette (tip diameter 30-50$\mu m$) under a fluorescent stereomicroscope (M165FC; Leica), and were washed 3 times by transferring to clean dishes. After the final wash, pure samples were aspirated in a small volume (1$\sim$3$\mu l$) and lysed in 47$\mu l$ XB lysis buffer (Picopure Kit, KIT0204; ThermoFisher) in a 200$\mu l$ PCR tube (Axygen), incubated for 30min at 40$^{\circ}$C on a thermal cycler and then stored at -80$^{\circ}$C. Detailed information on profiled samples are provided in Supplementary Table 2.

\subsection{RNA-seq}
Total RNA was extracted using the Picopure kit (KIT0204; ThermoFisher). Either 1$\mu l$ of $10^{-5}$ dilution of ERCC spike-in control (\#4456740; Life Technologies) or (number of sorted cells/50) * (1$\mu l$ of $10^{-5}$ dilution of ERCC) was added to the purified RNA and speed-vacuum concentrated down to 5$\mu l$ and immediately processed for reverse transcription using the NuGEN Ovation RNA-Seq System V2 (\#7102; NuGEN) which yielded 4$\sim$8$\mu g$ of amplified DNA. Amplified DNA was fragmented (Covaris E220) to an average of $\sim$200bp and ligated to Illumina sequencing adaptors with the Encore Rapid Kit (0314; NuGEN). Libraries were quantified with a KAPA Library Quant Kit (KAPA Biosystems) and sequenced on an Illumina HiSeq 2500 with 4 to 32-fold multiplexing (single end, usually 100bp read length, see Supplemental Table 2).

\subsection{RNA-seq analysis}
Adaptor sequences (AGATCGGAAGAGCACACGTCTGAACTCCAGTCAC for Illumina sequencing and CTTTGTGTTTGA for NuGEN SPIA) were removed from de-multiplexed FASTQ data using cutadapt v1.7.1 (http://dx.doi.org/10.14806/ej.17.1.200) with parameters “--overlap=7 --minimum-length=30”. Abundant sequences (ribosomal RNA, mitochondrial, Illumina phiX and low complexity sequences) were detected using bowtie2 \citep{Langmead_2012} v2.1.0 with default parameters. The remaining reads were mapped to the UCSC mm10 genome using STAR \citep{Dobin_2012} v2.4.0i with parameters “--chimSegmentMin 15 --outFilterMismatchNmax 3”. Mapped reads are quantified with HTSeq \citep{Anders_2014} using Gencode.vM13 \citep{Harrow_2012}.

\subsection{Annotations}
For reference annotations we used Gencode.vM13 \citep{Harrow_2012} downloaded from http://www.gencodegenes.org/, NCBI RefSeq \citep{Pruitt_2013} downloaded from the UCSC genome browser.

\subsection{Pan-neuronal genes}
Pan-neuronal genes are extracted as satisfying the following conditions: 1) mean neuronal expression level (NE)$>$ 20 FPKM, 2) minimum NE $>$ 5 FPKM, 3) mean NE $>$ maximum non-neuronal expression level (NNE), 4) minimum NE $>$ mean NNE, 5) mean NE $>$ 4x mean NNE, 6) mean NE $>$ mean NNE $+$ 2x standard deviation of NNE, 7) mean NE $-$ 2x standard deviation of NE $>$ mean NNE. 

\subsection{DI/SC/DM calculation}
To calculate DI, the following criteria were used to assign a "1" or "0" to each element in the difference matrix (DM): log fold change $>$ 2 and q-value $<$0.05. Q-values were calculated using the limma package including the voom method \citep{Law_2014}. To adjust the power to be similar across cell types, two replicates (the most recent two) are used for all cell types with more than two replicates. We have tried the same calculations with 3 replicates (using a fewer number of cell types) and obtained similar results (data not shown).
We also calculated DI and SC across four single-cell datasets: For \cite{Zeisel_2015} and \cite{Tasic_2016}, we used log fold change $>$ 1 and q-value $<$0.05 calculated using limma/voom method for differential gene expression. For \cite{Saunders_2018} and \cite{Zeisel_2018}, only cluster average expression was available, and log fold change $>$ 1 was defined as a criterion for differential expression.


%To calculate binary DI (bDI), the following DM criteria were used: expression levels of all the replicates in one of the cell types in the pair $<$ 1FPKM and expression levels of all the replicates in the other cell type in the pair $>$ 15FPKM, in addition to q-value $<$0.05. 

%To assess the extent of differentiation by alternative splicing, we calculate differentiation at the level of each splice branch. See Figure 3D for the definitions of a splice branch and of branch probability. For each branch, at each alternative splice site, we define each pair of cell types as "different" when 1) branch probabilities for all replicates in a group are less than 0.3 or greater than 0.7, and 2) both cell types in the pair have $>10$ reads reads at the alternative site. Condition 1) is justified by the bimodal distribution of branch probabilities shown in Figure 3E. Accumulating over all pairs creates a DM for each branch. We then combine all the branches using a logical "OR" to create a gene-level DM for each gene. If any branch distinguishes a pair of cell types, that pair is called "different" at the gene level. The gene-level DM has a value of "1" for pairs of cell types  distinguished by any of the branches belonging to that gene, and has a value of "0" for pairs of cell types not distinguished by any branch belonging to the gene. The number of pairs compared can differ, depending on the expression pattern of the gene, since branch probabilities can only be calculated for cell types that express the gene. This situation differs from that for DI or bDI (based on expression levels rather than splicing) since pairs of cell types can be distinguished even if one does not express the gene. Therefore, unlike DI and bDI which assume a fixed number of total pairs, we use DN (total number of pairs distinguished), rather than the fraction of pairs distinguished, to rank genes.

\subsection{NNLS/Random forest decomposition}
Single cell datasets deposited in NCBI GEO (GSE71585, \cite{Tasic_2016}; and GSE60361, \cite{Zeisel_2015}) were used for NNLS decomposition. Specifically the deposited count data were converted to TPM and used for comparison. The NeuroSeq dataset was quantified using RefSeq and featurecount \citep{Liao_2013} and converted into TPM. Subsets of genes common to all three datasets are then used for all further analyses. Since distributions of TPM values  differed between datasets, they were quantile normalized to an average profile generated from the NeuroSeq dataset. Since most genes in the single cell profiles exhibited noisy expression patterns, using the entire gene set for decomposition is not feasible. Therefore, we selected for decomposition the genes deemed most informative for distinguishing cell classes based on ANOVA across cell classes. However, simply taking the top ANOVA genes lead to highly biased gene selection since some cell types exhibited much larger transcriptional differences than others (e.g. many ANOVA selected genes were specific to microglia). We therefore selected genes so as to minimize the overlap between the cell types distinguished. Beginning with the highest ANOVA gene (highest ANOVA F-value), genes were selected only if their DM (Differentiation Matrix defined in Figure 2) differed from those previously selected, defined with a Jaccard index threshold  of 0.5. We chose 300 genes from each dataset, yielding a total of 563 genes when all three sets were combined. This gene set was then used for all decompositions. Decompositions were performed on average profiles created by summing NeuroSeq replicates or by summing single-cell profiles using cluster assignments provided by the authors. NNLS was implemented using the Python scipy library (http://www.scipy.org). 

For Random forest, implementation in the Python scikit-learn library \citep{scikit-learn} was used. 

\subsection{ATAC-seq}
7 cell types, Purkinje and granule cells from cerebellum, excitatory layer 5, 6 and entorhinal pyramidal cells from cortex, excitatory CA1, or CA1-3 pyramidal cells from hippocampus, labeled in mouse lines P036, P033, P078, 56L, P038, P064, and P036 respectively \citep[all from][]{Shima_2016} were profiled with ATAC-seq. They were FACS sorted to obtain $\sim$20,000 labeled neurons. ATAC libraries for Illumina next-generation sequencing were prepared in accordance with a published protocol \citep{Buenrostro_2013}. Briefly, collected cells were lysed in buffer containing 0.1\% IGEPAL CA-630 (I8896, Sigma-Aldrich) and nuclei pelleted for resuspension in tagmentation DNA buffer with Tn5 (FC-121-1030, Illumina). Nuclei were incubated for 20-30 min at 37$^{\circ}$C. Library amplification was monitored by real-time PCR and stopped prior to saturation (typically 8-10 cycles). Library quality was assessed prior to sequencing using BioAnalyzer estimates of fragment size distributions looking for a ladder pattern indicative of fragmentation at nucleosome intervals as well as qPCR to determine relative enrichment at two housekeeping genes compared to background (specifically the TSS of \textit{Gapdh} and \textit{Actb} were assessed relative to the average of three intergenic regions). For sequencing, Illumina HiSeq 2500 with 2 to 4-fold multiplexing and paired end 100bp read length was used. In addition to ATAC-seq, RNA-seq was performed on replicate samples of $\sim$2,000 cells collected in a similar way, and library prepared using the same method described above.

\subsection{ATAC-seq analysis}
Nextera adaptors (CTGTCTCTTATACACATCT) were trimmed from both ends from de-multiplexed FASTQ files using cutadapt with parameters "-n 3 -q 30,30 -m 36". Reads were then mapped to UCSC mm10 genome using bowtie2 \citep{Langmead_2012} with parameters "-X2000 --no-mixed --no-discordant". PCR duplicates were removed using Picard tools (http://broadinstitute.github.io/picard, v2.8.1) and reads mapping to mitochondrial DNA, scaffolds, and alternate loci were discarded. BigWig genomic coverage files were generated using bedtools \citep{Quinlan_2010} and scaled by the total number of reads per million. For reproducible peaks, liberal peaks were called using HOMER (v4.8.3) \citep{Heinz_2010} with parameters "-style factor -region -size 90 -fragLength 90 -minDist 50 -tbp 0 -L 2 -localSize 5000 -fdr 0.5" and filtered using the Irreproducibility Discovery Rate (IDR) in homer-idr (http://github.com/karmel/homer-idr.git) with parameters "--threshold 0.05 --pooled-threshold 0.0125". Peak counts and peak patterns were then quantified using bedtools.

%%%%%%%%%%%%%%%%%%%%%%%%%%%%%%%%%%%%%%%%%%%%%%%%%%%%%%%%%%%%%%%%%%%%%%%%%%%%%%%%

\subsection{Inserted segments}
The multiz alignments downloaded from the UCSC genome browser \citep{Kent_2002} was used to calculate inserted segments in human or mouse. By comparing closely related species, human (hg38) vs. chimp (panTro6) or mouse (mm10) vs. rat (rn6), candidate segments inserted into human (or mouse) are extracted. By using another closely related species as a common ancestor, gorilla (gorGor5), chinese hamster (criGri1), respectively for human/chimp and mouse/rat, segments absent in chimp and gorilla (or absent in rat/chinese hamster) are called insertion in human (or mouse), and segments absent in chimp but present in gorilla (or absent in rat but present in chinese hamster) are called deletion in chimp (or rat). 

\subsection{ME fitting}
Repeat annotations for mouse mm10 genome as detected by RepeatMasker \citep{repeatmasker} with Repbase \citep[ver. 20140131][]{Bao_2015} were used. Only repeat families with number of instances$>$200 are included. For individual repeats, only those with number of instances$>$50 are included. For repeats in the "Simple repeat" class, only those with number of instances$>$1000 are included. Repeat scores are calculated as described in Figure 5C using Gencode.vM13. Only genes with non-zero repeat scores are used for fitting. For fitting expression level (rank) by repeat score, a regularized version of linear regression, Ridge regression, implemented in the Python scikit-learn library \citep{scikit-learn} was used.

\subsection{Anatomical region abbreviations}
Region abbreviations: AOBmi, Accessory olfactory bulb, mitral layer; MOBgl, Main olfactory bulb, glomerular layer; PIR, Piriform area; COAp, Cortical amygdalar area, posterior part; AOBgr, Accessory olfactory bulb, granular layer; MOBgr, Main olfactory bulb, granular layer; MOBmi, Main olfactory bulb, mitral layer; VISp, Primary visual area; AI, Agranular insular area; MOp5, Primary motor area, layer5; VISp6a, Primary visual area, layer 6a; SSp, Primary somatosensory area; SSs, Supplemental somatosensory area; ECT, Ectorhinal area; ORBm, Orbital area, medial part; RSPv, Retrosplenial area, ventral part; ACB, Nucleus accumbens; OT, Olfactory tubercle; CEAm, Central amygdalar nucleus, medial part; CEAl, Central amygdalar nucleus, lateral part; islm, Major island of Calleja; isl, Islands of Calleja; CP, Caudoputamen; CA3, Hippocampus field CA3; DG, Hippocampus dentate gyrus; CA1, Hippocampus field CA1; CA1sp, Hippocampus field CA1, pyramidal layer; SUBd-sp, Subiculum, dorsal part, pyramidal layer; IG, Induseum griseum; CA, Hippocampus Ammon’s horn; PVT, Paraventricular nucleus of the thalamus; CL, Central lateral nucleus of the thalamus; AMd, Anteromedial nucleus, dorsal part; LGd, Dorsal part of the lateral geniculate complex; PCN, Paracentral nucleus; AV, Anteroventral nucleus of thalamus; VPM, Ventral posteromedial nucleus of the thalamus; AD, Anterodorsal nucleus; RT, Reticular nucleus of the thalamus; MM, Medial mammillary nucleus; PVH, Paraventricular hypothalamic nucleus; PVHp, Paraventricular hypothalamic nucleus, parvicellular division; SO, Supraoptic nucleus; DMHp, Dorsomedial nucleus of the hypothalamus, posterior part; ARH, Arcuate hypothalamic nucleus; PVHd, Paraventricular hypothalamic nucleus, descending division; SCH, Suprachiasmatic nucleus; LHA, Lateral hypothalamic area; SFO, Subfornical organ; VTA, Ventral tegmental area; SNc, Substantia nigra, compact part; SCm, Superior colliculus, motor related; IC, Ingerior colliculus; DR, Dorsal nucleus raphe; PAG, Periaqueductal gray; PBl, Parabrachial nucleus, lateral division; PG, Pontine gray; LC, Locus ceruleus; CSm, Superior central nucleus raphe, medial part; AP, Area postrema; NTS, Nucleus of the solitary tract; MV, Medial vestibular nucleus; NTSge, Nucleus of the solitary tract, gelatinous part; DCO, Dorsal cochlear nucleus; NTSm, Nucleus of the solitary tract, medial part; IO, Inferior olivary complex; VII, Facial motor nucleus; DMX, Dorsal motor nucleus of the vagus nerve; RPA, Nucleus raphe pallidus; PRP, Nucleus prepositus; CUL4,5mo, Cerebellum lobules IV-V, molecular layer; CUL4,5pu, Cerebellum lobules IV-V, Purkinje layer; PYRpu, Cerebellum Pyramus (VIII), Purkinje layer; CUL4,5gr, Cerebellum lobules IV-V, granular layer; MOE, main olfactory epithelium; VNO, vemoronasal organ.








