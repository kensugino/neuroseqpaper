% Abstract < 150 words 
\begin{abstract} 
The diversity of mammalian neuronal cell types is well documented, but the principles underlying this diversity are poorly understood. Here we present RNA-seq data from 193 genetically identified cell types. To better quantify expression diversity we introduce two metrics that distinguish information content from signal-to-noise-ratio. We find the homeobox transcription factors family contains the highest information about cell types with unusually low noise. Highly informative effector genes tend to be long, and are enriched for synaptic genes, are repressed in non-neuronal cells, and contribute disproportionately to neuronal diversity. Consistent with needing more diverse regulation, genome accessibility measurements reveal that long genes have more candidate regulatory elements arrayed in more distinct patterns. This may have implications for the evolution of complex nervous systems, as species with more complex brains have longer introns in orthologous neuronal genes presumably via insertion of mobile elements that evolve into cell type-specific regulators.

%% currently 142 words 
\end{abstract}
