% Abstract < 150 words (elife)
\begin{abstract} 
The mammalian nervous system is constructed of many cell types, but the principles underlying this diversity are poorly understood. To assess brain-wide transcriptional diversity, we sequenced the largest collection of genetically/anatomically identified neuronal classes. Using improved expression metrics that distinguish information content from signal-to-noise-ratio, we found that expression of homeobox transcription factors is tightly controlled and their binary expression pattern can distinguish 98\% of cell types profiled. Genes that contribute the most to neuronal diversity tend to be long and enriched in factors specifically involved in neuronal function. Genome accessibility measurements reveal that long genes have more candidate regulatory elements with distinct patterns. These elements frequently overlap interspersed repeats (mobile elements) and the pattern of repeats is predictive of gene expression. This suggests that regulatory potential of long genes have been enhanced by mobile element insertion. Thus, elongation of neuronal genes by mobile elements may be an evolutionary force to expand brain complexity.
%% currently 150 words 
\end{abstract}


