\section{Introduction}

% - remove assembler part
% - add homeobox
% - add regional difference etc.

The extraordinary diversity of vertebrate neurons has been appreciated since the proposal of the neuron doctrine (Cajal, 1995). Typically, this diversity is characterized by neuronal morphology, physiology, molecular expression, and circuit connectivity. The exact number of neuronal cell types remains unknown, but estimates of 40-60 have been provided for the retina (Macosko et al., 2015; Masland, 2004) and for mouse cortex (Tasic et al., 2016; Zeisel et al., 2015). If similar numbers are discovered in most brain regions, the number could be in the thousands or more. Describing the cell types in the brain is a major and necessary goal of neuroscience.

Although neuronal diversity has long been recognized, the question of how this diversity arises has just started to be asked (Arendt, 2008; Muotri and Gage, 2006). The number and diversity of neuronal cell types are almost certainly related to the brain complexity and behavioral flexibility. For example, the primate cortex and thalamus contain many more regions, nuclei and laminae than the homologous structures in rodents, which in turn have more complex forebrains than more primitive vertebrates. Therefore, investigation of the origin of neuronal diversity can lead to the origin of the differences in behavioral flexibility between species. 

The phenotypic differences between neuronal cell types must arise from complex interactions between their expressed proteins, which in turn must reflect differences in transcription, splicing, translation, posttranslational modification and localization. Although a complete account of neuronal diversity will require addressing all of these biological processes, numerous studies suggest that significant diversity occurs already at the transcriptional level (Levine and Tjian, 2003) and that splicing can be another important source of transcriptome diversity (Andreadis et al., 1987). Here we provide an extensive analysis of these sources of diversity at the transcriptional level for a large number of cell types in the mouse brain.

To obtain transcriptional profiles of diverse cell types, we sorted fluorescently labeled neurons from micro-dissected mouse brain regions  (Hempel et al., 2007) and performed deep RNA sequencing (RNA-seq). Using this approach, we determined the complete gene expression profiles for 174 neuronal cell types and 15 non-neuronal cell types.

To reveal the sources of the diversity, we developed a new junction- and coverage-based RNA-seq assembler and new methods to quantify differential expression and splicing. By quantifying gene's ability to distinguish cell types on the basis of both transcription and splicing, we identified that long genes contribute disproportionately to neuronal diversity. Interestingly, genes selectively expressed in neurons are also biased toward long genes. These long genes are enriched in cell adhesion and ion channel genes, indicating they convey circuit level differences (connectivity and input/output function). 

Recently, long genes have been recognized to play roles in multiple diseases of the nervous system (Gabel et al., 2015; Sugino et al., 2014; Wei et al., 2016; Zylka et al., 2015). They are also metabolically expensive to produce (Castillo-Davis et al., 2002). Then, why neuronal genes are biased toward long genes? Based on our finding that long genes convey transcriptional diversity, we propose that benefits to brain complexity may outweigh the metabolic expense and propensity to disease accompanying their neuronal expression. 

The data and tools presented here also provide valuable resources for further exploration of transcriptome regulation and diversity, identification of marker genes, and gene expression and splicing look up tables to explore function within given cell types. For easy exploration of the dataset, a web site is provided (neuroseq.janelia.org). 