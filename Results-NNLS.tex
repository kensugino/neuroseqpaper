\subsection{Comparison to Single Cell datasets}
A potential problem with profiling pools of sorted neurons is that two distinct types of neurons may be inadvertently profiled together if they are both labelled in the same brain region of the same strain. Single cell (SC) RNA-seq has recently emerged as a viable method for profiling cellular diversity in the nervous system and elsewhere, that does not suffer from this limitation. Profiles of cell types in SC studies are obtained by clustering and then summing cellular profiles. In order to assess the relative cellular homogeneity of our sorted samples we compared their profiles to cluster profiles from SC studies. We focused on neuronal and non-neuronal cell types in the neocortex, since two recent SC studies were available for making this comparison (Tasic et al. 2015; Zeisel et al. 2015). Transcriptional homogeneity was computed by decomposing NeuroSeq profiles into SC cluster profiles using non-negative least squares (NNLS). As a check on the validity of the procedure we decomposed summed profiles of multiple SC clusters obtained from a single transgenic lines in Tasic et al. The resulting coefficients (Fig. 2 Supp. 1, left) very closely reproduced the reported fractions of cells of each type obtained from each transgenic strain (Fig. 2 Supp. 1, right).

For each sorted cell type in the NeuroSeq dataset the procedure identifies the weight (coefficient) of each component cluster cell type in the SC datasets (Fig. 2A). Fully blue columns represent cell types profiled in the Tasic and Zeisel studies that could not be matched to NeuroSeq cell types. These included L4 neurons, VIP interneurons and oligodendrocytes which were not profiled in our study. 