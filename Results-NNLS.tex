\subsection{Comparison to single cell datasets}
Pools of sorted neurons may be heterogeneous if multiple neuronal subtypes are labeled in the same brain region of the same strain. Single cell RNA-seq (SCRS) has recently emerged as a viable method for profiling cellular diversity, that does not suffer from this limitation. However, since profiles of cell types in SCRS studies are obtained by clustering individual noisy cellular profiles, inaccuracies can arise from misclustering or overclustering. In order to assess the relative cellular homogeneity of our sorted samples we compared their profiles to cluster profiles from SCRS studies. We focused on neuronal and non-neuronal cell types in the neocortex, since two recent SCRS studies profiled this region \cite{Tasic_2016,Zeisel_2015}. Transcriptional homogeneity was computed by decomposing NeuroSeq profiles into SCRS cluster profiles using non-negative least squares (NNLS). We performed multiple checks on the validity of the procedure (see Methods and Figure 2 Supplementary Figures). 

For each sorted cell type in the NeuroSeq dataset the procedure identifies the weight (coefficient) of each component cluster cell type in the SCRS datasets (Fig. 2A). As expected, cell types present in the SCRS studies, but not profiled in the NeuroSeq study, (e.g. L4 neurons, VIP interneurons and oligodendrocytes), were not matched (purely blue columns in Figure 2A). Other cell types matched perfectly to a single SCRS cell type (microglia, astrocytes, ependyma) or matched to more than one, implying heterogeneity in the sorted profiles or poor separation of the SCRS profiles. We measured the "purity" of the decomposition (fractional match to the highest coefficient). The purity scores for the decomposition of NeuroSeq cell types by the two SCRS datasets are higher than those obtained for SCRS cell clusters decomposed by the other SCRS data set (Figure 2B and Fig2 Supplement 3). This implies that although sorted sample heterogeneity may occur, it is comparable (or less) than the inaccuracies introduced by clustering. Profiles with imperfect matches usually matched closely related cell types. For example, the NeuroSeq Pvalb interneuron group matched one or two of the SCRS Pvalb-positive interneuron clusters, and layer 2/3 (L2/3) pyramidal neurons matched SCRS L2/3 clusters, or an adjacent cluster in L4 (Tasic: L4 Arf5). The spread of coefficients repeatedly involved the same few SCRS cell clusters (e.g. columns Tasic: L5b Tph2 and L5b Cdh13; Zeisel: S1PyrL5 and S1PyrL6), which could occur if these clusters are not well separated. A cross-validation procedure confirmed that some clusters are poorly separated (Figure 2 Supplement 4).   

To directly assess cluster separability, we calculated the relative distances separating each pair of profiles in each data set (Figure 2D). Most SCRS clusters within layers or within interneuron subtypes are overlapping. In contrast nearly all NeuroSeq profiles are well separated. The same data are plotted in the lower row using an expanded range (0-10) to illustrate separability over a larger dynamic range. In summary, NeuroSeq sorted cell types are as homogeneous as SCRS clusters but have much higher separability. 

