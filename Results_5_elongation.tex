\subsection{Gene length correlates with nervous system complexity}
Vertebrate introns are known to have undergone significant expansion through multiple rounds of transposition \cite{Grishkevich_2014}, and this process is known to have accelerated during mammalian evolution, and more specifically, during primate evolution \cite{Friedli_2015}. Recent estimates suggest that half to two thirds of the human genome may be derived from mobile elements \cite{Friedli_2015}[#wrong ref?No, correct]. To map genomic insertions potentially contributing to the elongation of genes, we analyzed the genomic sequence differences between the closely related primate genomes of human and chimp and the closely related rodent genomes of mouse and rat. In each case, we used a third species (gorilla for human and chinese hamster for mouse) to identify ancestral sequences, allowing more recent insertions in each lineage to be identified. When the histograms of the lengths of the inserted intervals are plotted, two peaks are apparent (Figure 5A). The shorter peak corresponds to Alu insertions and longer peak corresponds to L1 insertions. Overall, 88\% of the inserted base pairs (of size between 100bp and 100kbp) in human and 94\% of those in mouse overlap annotated mobile elements {Hubley, 2016 #98}. Mobile elements are inserted both into intergenic regions as well as gene-bodies (Figure 5B), albeit the insertion rate are lower in gene-bodies. These and analyses from other groups \cite{Grishkevich_2014} [#fix ref Chimpanzee, 2005 #79}] indicate that genes are elongated by insertion of specific mobile elements during evolution.

Since long genes have a greater number of candidate regulatory elements, as indicated by more ATAC-peaks, we asked whether these can originate from mobile elements. As shown in Fig. 5C, 53\% of the ATAC peaks overlap known repeats and this number increases to 93\% when only newly inserted segments are considered, indicating that MEs may carry regulatory functions. To determine if MEs within long genes shape their diverse neuronal expression patterns, we asked whether the complement of MEs can be used to predict part of those expression patterns. We fit gene expression levels with counts of individual MEs within and surrounding each gene (Figure 5D left). The $R^2$ values for each cell type calculated using test genes (20\%) not used for fitting (Figure 5D right, blue) are much larger than expected by chance (Figure 5D right, green/red/orange). If counts and genes are shuffled (green) cross-validated $R^2$ values drop below 0. We found that if the length of the gene is retained in the shuffling control (orange, red) the $R^2$ values drop to about 1/3 of those in the original fitting. This reflects the fact that gene length is highly correlated with expression (Figure 4 Supplement 1[#fix supp]; c=0.418: mean Pearson’s r between log gene length and expression rank) and with the presence of some repeats, such as SINEs, which are highly correlated with both gene length (c=0.841) and expression (Figure 4 Supplement 2[#fix supp]; mean c=0.454). Although many other factors likely contribute to shaping gene expression across neuronal cell types, consistent with the fact that $R^2$ values were below 0.3, the regression analysis suggests that MEs make a highly significant contribution to expression patterns. These results suggest that the elongation of neuronal effector genes by insertions of mobile elements have endowed them with increased capacity for differential expression, permitting enhanced neuronal diversity. 

The above results indicate that gene length is an important contributor to gene expression diversity across cell types. Cell type diversity is a hallmark of the nervous system, and is thought to be critical for its ability to generate complex behavioral outputs. Thus gene length and the mechanisms that drive lengthening over time are potential contributors to the evolution of nervous system complexity. Interestingly, the expression of long genes is not uniform across brain regions, but is highest in the evolutionary newer forebrain and is lower in the older brainstem and hypothalamus (Figure 5E). Non-neuronal cell types expressed only 1/2 to 1/5 as many long genes as neuronal cell types (blue bars in Figure 5E). This was true even for non-dividing cell types like myocytes and largely non-dividing tissues like the heart (data not shown). 

The involvement of gene length in cell type diversity predicts that species with longer genes will have a larger, more diverse set of neuronal cell types, permitting construction of more complex neural circuits. Consistent with this, we find that the distribution of gene lengths for the best annotated species is broadly correlated with nervous system size and complexity (Figure 5F). This is also true of neuronal genes, (defined in the mouse), which also increase in length with increased nervous system complexity (data not shown). 

In summary, genes are elongated by mobile element insertions which carry regulatory sites, and species with complex brain have longer sets of genes, consistent with the idea that increased capacity of neuronal genes to be differentially expressed provided by mobile element insertions contributes to neuronal diversit.




