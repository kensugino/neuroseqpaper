\section*{Introduction}

The extraordinary diversity of vertebrate neurons has been appreciated since the proposal of the neuron doctrine \citep{Cajal_1888}. Classically, this diversity was characterized by neuronal morphology, physiology, and circuit connectivity, but increasingly, defined genetically through driver and reporter strains \citep{Gong_2003, Madisen_2009, Taniguchi_2011, Shima_2016} or genomically by their genome-wide expression profiles. The first genome-wide studies of mammalian neuronal diversity employed \textit{in situ} hybridization \citep{Lein_2006} or microarrays \citep{Sugino_2005, Doyle_2008}, while more recent studies have utilized advances in single cell (SC) RNAseq \citep{Zeisel_2015, Zeisel_2018, Tasic_2016, Tasic_2018, Paul_2017}. In theory, SC RNAseq can be applied in an unbiased fashion to discover all cell types that comprise a tissue, but manipulation of these cell types to better understand their biological composition and function often require the use of genetic tools such as mouse driver strains. Differences in techniques for cell isolation, library preparation or clustering have not yet led to a consensus view of the number or identity of the neuronal cell types comprising most parts of the mouse nervous system. Furthermore, the relationship between cell populations defined transcriptionally and those that can be specified genetically and anatomically using existing strains has received far less attention (though see \citealt{Tasic_2018}).    
 
Here we attempt to strengthen the link between genomically and genetically defined cell types in the mouse brain by performing RNA-seq on a large set of genetically identified and fluorescently labeled neurons from micro-dissected brain regions. In total, we profiled 179 sorted neuronal populations and 15 non-neuronal populations. Because each sample of sorted cells may contain more than one ”atomic” cell type, we refer to these as genetically- and anatomically-identified cell populations (GACPs). To assess homogeneity, we quantitatively compared our sorted GACPs to publicly available single cell datasets, which revealed a comparable level of homogeneity, but a much lower level of noise in the sorted population profiles.

Although neuronal diversity has long been recognized, the question of how this diversity arises has not been addressed sufficiently in a genomic context \citep{Arendt_2016, Muotri_2006}. We identify two different sets of genes that distinguish GACPs based on the robustness or pattern of their expression differences. The most robust expression differences are those of homeobox transcription factors. These genes also have the lowest transcriptional noise suggesting differential chromatin regulation. Chromatin accessibility measurements reveal that the promoters and gene bodies of these genes are indeed more closed. In contrast, the genes capable of distinguishing the largest numbers of GACPs are neuronal effector genes like receptors, ion channels and cell adhesion molecules. Interestingly, genes defined by the robustness and patterns of their expression differences also differ in their transcript length. Genes with robust, low noise expression tend to be shorter than the genes with the greatest capacity to distinguish populations, which tend to be longer.

Here we provide important new resources for mapping brain cell types including a large set of low-noise profiles from genetically identified neurons, anatomical maps of their distributions, and a method to compare and contextualize single cell RNA-seq datasets. We implement a novel strategy to mine information from large surveys of cell types, and demonstrate the utility of this strategy in generating specific biological insights into the genes contributing to neuronal diversity.

%To rigorously assess diversity in our dataset we adopt a well-established approach from the field of ecology \citep{Simpson_1949}. Using this information theoretic approach we introduce two simple metrics that separate key features of the data. Signal contrast (SC) is a signal-to-noise ratio that (unlike ANOVA) is not sensitive to differences in information content. Differentiation index (DI) is a measure of information on cell types contained in gene expression pattern closely related to mutual information between cell types and expression levels. We find these two metrics, DI and SC, have improved performance, and provide novel insights about the types of genes contributing to neuronal diversity.

%The finding that long genes contribute disproportionately to neuronal diversity is consistent with the previous finding that long genes are preferentially expressed in the brain. ATAC-seq reveals that long genes also contain a larger number of candidate regulatory regions that are also arrayed in more diverse patterns across neuronal cell types than in short genes. We hypothesize that this may permit increased regulatory complexity, thereby enhancing neuronal diversity. %However, it raises a conundrum. Long genes have documented costs with respect to metabolism and genome integrity, and yet they are widely expressed in the brain. %Our findings suggest that their contribution to cellular diversity is a possible benefit counterbalancing the costs. Investigating further, we find that long genes have lengthened through insertions of mobile elements and contain a larger number of candidate regulatory regions identified by ATAC-seq that are also arrayed in more diverse patterns across neuronal cell types than in short genes. This suggests that mobile element insertions have provided genetic fodder to increase the diversity of transcriptional regulation, thus facilitating cellular diversity. Over evolution, brains that have become more complex and more reliant on cellular diversity maintain long genes that continue to elongate as the genes themselves age. In contrast, we find low noise genes remain short, likely protected by closed chromatin biasing against insertion events and purifying selection due to their fundamental roles in early development.
%[# Stick somewhere in above: that high SC genes (i.e. homeobox), though low in DI individually has high combined DI.]


%We have curated these reproducible and precise expression profiles to serve as a look-up table for linking single cell and cell type expression profiles to genetic strains in which they can be repeatedly accessed. 

%Cell types are typically identified by performing differential expression analyses. Standard differential expression methods focus on signal variance but are influenced by both information content and robustness of differential expression. We introduced two simple metrics to separate out these features of the data. Signal contrast (SC) is a signal-to-noise ratio that (unlike ANOVA) is not sensitive to differences in information content. Differentiation index (DI) is a measure of information content closely related to mutual information. Using these metrics, we identify homeobox transcription factors (TF) as the gene family with the lowest noise and highest ability to distinguish cell types and use these and other TFs to construct a compact “code” for profiled neuronal cell types. We find that the effector genes carrying the most information about cell types are synaptic genes like receptors, ion channels and cell adhesion molecules. Interestingly, a common feature of these genes is their long genomic length, reflecting the increased number and length of their introns. 

%Long genes such as these have recently been recognized as sites of genomic instability \cite{Wei_2016}, and as contributors to multiple nervous system diseases \cite{Sugino_2014,Gabel_2015,Zylka_2015}. Given these hazards and the fact that long genes are metabolically expensive to produce \cite{Castillo_Davis_2002}, why should they be so prominent in the brain?

%Our ATAC-seq results indicate that long genes contain a larger number of candidate regulatory regions which are arrayed in more diverse patterns than found in short genes, suggesting the longer length of the genes may permit increased regulatory complexity.

%Our results suggest their length may permit increased regulatory complexity, with long genes containing a larger number of candidate regulatory regions identified by ATAC-seq arrayed in more diverse patterns across neuronal cell types than found in short genes. 

%Moreover, these long genes are elongated during evolution by insertions of mobile elements and a large portion of the candidate regulatory regions identified by ATAC-seq overlap with these mobile elements. Thus, the increased length of neuronal genes may provide a platform for evolution to fine-tune gene expression and thus diversify the cell types of the nervous system.
