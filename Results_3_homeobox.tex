\subsection{Homeobox TFs have the highest SNRs and form a combinatorial code for cell types}
FCR, like SNR, is a ratio between signal and noise, and so can reflect high expression levels in most ON cell types (high signal), low expression levels in most OFF cell types (low noise), or both. Homeobox genes are not among the most abundantly expressed genes. Their average expression levels ($\sim$30 FPKM) are significantly lower than, for example, those of neuropeptides ($\sim$90 FPKM). This suggests that the high FCR of homeobox TFs depend more on low noise than high signal. In fact, most homeobox TFs have uniformly low expression in OFF cell types (Figure 5A top). We quantified this "OFF noise" for all genes and found that homeobox genes are enriched among genes that have both low OFF noise and at least moderate ON expression levels (red dashed region in Figure 5B). This enrichment was also observable in two of the single cell datasets encompassing multiple brain regions (Figure 5 Supplement 2). 

Since tight control of expression may reflect closed chromatin, we measured chromatin accessibility using ATAC-seq (see methods). As expected, compared to high-noise genes (Figure 5C bottom), genes with low OFF noise were more likely to have fewer, smaller peaks within the vicinity of their transcription start site (TSS) and gene body (Figure 5C top, Figure 5D), consistent with the idea that their expression is controlled at the level of chromatin accessibility. Functionally, the tight control of homeobox TF expression levels may reflect their known importance as determinants of cell identity, and the fact that establishing and maintaining robust differences between cell types may require tight ON/OFF regulation rather than graded regulation.

The ability of gene families to provide information about cell types reflects both how informative individual family members are, and the relationships between them. If the information across family members is independent, the overall information is increased relative to the case in which multiple members contain redundant information. This aspect of "family-wise" information is not captured by "gene-wise" metrics like mean DEF, or by enrichment analysis (Figure 4B,C). One means of capturing the additive, non-redundant information within a gene family is to measure its ability to separate cell types in terms of expression distances defined by member genes. This analysis (Figure 5E) reveals that homeobox TFs provide the greatest separability between cell types among PANTHER families. This is consistent with the observation that their expression patterns across cell types show low correlation because each tends to be expressed in only a small number of distinct cell types (Figure 5 Supplement 1A). Related to this, we found that the homeobox family can distinguish 97\% of the cell type pairs, suggesting these TFs form a combinatorial code for the cell types profiled (Figure 5 Supplement 1B). 

In summary, we found that homeobox genes are tightly controlled at the chromatin level and in aggregate, have the greatest ability to separate cell types. This suggests that, similar to other tissues \citep{Kratsios_2017,Zheng_2015,Dasen_2009,Philippidou_2013}, homoebox TFs play an important role in specifying cell types in the brain.





