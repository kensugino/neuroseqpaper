\subsection{Long genes contribute disproportionately to neuronal diversity}

Gene expression is known to be controlled at the chromatin level, by a combination of TF at the promoter region, or by enhancer/repressor elements. We reasoned that increased ability of high DI genes to be differentially expressed may be dependent on the relevant number of these cis-regulatory elements. Since these elements need to be isolated to each other in the genome, we inferred that longer genes may have higher capacity to be differentially expressed. This is in fact the case and longer genes have on average higher DIs (Fig.6A).

Recently, long genes have been found to be enriched in neurons and are frequently linked to developmental disorders of the CNS (Gabel et al., 2015; Sugino et al., 2014; Wei et al., 2016; Zylka et al., 2015). Consistent with these, in our dataset we observed that neurons express more long genes compared to non-neurons (Fig.6B). Since neuronal genes are often repressed by NRSE in non-neurons, we wondered whether REST targets are long. When length of the genes nearest to computationally detected REST elements are plotted, we observed that they are indeed biased toward long genes (Fig.6C). When these REST targets are removed from neuronally expressed genes, the length distribution looks similar to that of non-neurons (Fig.6D).
