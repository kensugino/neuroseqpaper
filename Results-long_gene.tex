\subsection{Long genes contribute disproportionately to neuronal diversity}
%% Incorporate both gene level expression (binary/graded) and splice

Recently, long genes have been found to be enriched in neurons and are frequently linked to developmental disorders of the CNS (Gabel et al., 2015; Sugino et al., 2014; Wei et al., 2016; Zylka et al., 2015). In our dataset, we found that neuronal cell types express more long genes than non-neuronal cell types (Figure 6A). This bias was absent for muscle cells, which are also post-mitotic, and was strongest in the cortex and hippocampus (Figure S6A). Since fewer non-neuronal cell types were included in our dataset, we repeated the calculation while varying the number of neuronal and non-neuronal samples included (Figure S6B-E). This varied the number of neuronal and non-neuronal specific genes but the long gene bias of neuronal specific genes was present in all cases. Furthermore, the bias was also present when comparing all genes expressed in neuronal vs. non-neuronal samples, regardless of whether or not they were restricted to one group or the other (Figure S6A). 
Neuron-specific genes (expressed in some neurons, but not in non-neuronal cell types in this dataset) were divided into long or short neuronal genes based on whether they were longer or shorter than 100kbp. The 569 long and 1620 short, neuron-specific genes in our dataset are both enriched for ion channels and transporters. In addition, long neuron-specific genes are enriched for cell adhesion molecules, apolipoproteins and metalloproteases, while short neuronal specific genes are enriched for GPCRs, homeobox transcription factors and peptide hormones (Figure 6B).
Long genes not only distinguished neurons from non-neurons, they also distinguished neuronal cell types from each other, as neuron-specific long genes have higher average DIs for both gene expression and splicing (Figure 6C). Consistent with this, DI is strikingly length dependent (Figure 6D). DI calculated separately for distinctions among non-neuronal cell types was not length dependent (not shown).
In RNA-seq experiments, longer genes have more reads and hence reduced variance (Oshlack and Wakefield, 2009); however this does not account for the observed length dependencies of DI, since they were present even when using reads only from the 3' 1000 bp of each gene (Figure S6F-I). Thus, long genes contribute disproportionately to the diversification of neuronal gene expression.