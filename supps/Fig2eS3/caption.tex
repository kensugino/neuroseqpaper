\textbf{Self decompositions by NNLS}.
 Each dataset is randomly divided into two groups and one is used to decompose the other. Coefficients matrix with perfect decomposition would be diagonal. Non-diagonal elements indicate limitation of the decomposition method due to having a subset of cell groups too similar to each other. \textbf{(A-C)} Heatmaps illustrate NNLS coefficients for subsets of samples in each dataset. Column order is same as row order. \textbf{(A)} 25 neocortical samples from \cite{Tasic_2018} \textbf{(B)} 25 neocortical samples from \cite{Zeisel_2018} \textbf{(C)} 28 neocortical samples from present study. \textbf{(D)} Mean purity scores (as defined in Figure 2) for cross-validation were comparable in each dataset.
 
 %For example, pairs of populations identified in layer 2/3 of two different regions in the same strain (AI.L23_glu_P157 / ORBm.L23_glu_P157) or by retrogradely labeling cells in the same layer and region from two different targets (SSp.L23_glu_M1.inj / SSp.L23_glu_S2.inj and SSp.L5_glu_BPn.inj / SSp.L5_glu_IRT.in) are hard to distinguish.\textbf{(D)} Purity scores (similar to Figure 2C) for the cross-validated NNLS decomposition of each Tasic et al. 2016 cluster.  