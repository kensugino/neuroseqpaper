\textbf{ME insertions and gene properties}\textbf{ (A), (B)} Distributions of length of mobile element insertions (A) and gene lengths (B) for low OFF noise genes (Figure 3B red dashed region) and high OFF noise genes (Figure 3B blue dashed region). Red lines are medians and whiskers indicate 1.5 Inter-quartile range. (p-values indicated are from Student’s t-test). \textbf{(C)} The percentages (in length) of ME insertions into genes are plotted against gene length. Each dot represents average of 2000 human or mouse genes in each bin, each bin defined by sorted gene length. Long genes have more insertions. This may reflect two effects: first, a subset of shorter genes may have inaccessible chromatin, as shown in Figure 3 and panels A,B of this figure. Second, there may be negative selection for insertions in or near exons, coupled to the fact that exons make up a progressively smaller fraction as gene length increases (Figure 4 Supplement 2A) \textbf{(D)} Distribution of gene length of genes with top 100 DI and top 100 SC. High SC genes are on average about 5 times shorter. \textbf{(E)} Overlap between major repeat classes and ATAC peaks for all peak sequence and for inserted sequences (as defined in Figure 5).