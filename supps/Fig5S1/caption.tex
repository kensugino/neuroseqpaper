\textbf{(Figure 5 Supplement 1)}\textbf{(A),(B)} Distributions of length of mobile element insertions (A) and gene lengths (B) for low OFF noise genes (Figure 3B red dashed region) and high OFF noise genes (Figure 3B blue dashed region). Red lines are medians and whiskers indicate 1.5 Inter-quartile range. (***:p<1e-100, Student’s t-test). \textbf{(C)} The total length of the genomic insertions histogrammed in Figure 5 are plotted as a fraction of gene length. Each dot represents 2000 human or mouse genes of the average length indicated. Long genes have more insertions. This may reflect two effects: first, a subset of shorter genes may have inaccessible chromatin, as shown in Figure 3 and panels A,B of this figure. Second, there may be negative selection for insertions in or near exons, coupled to the fact that exons make up a progressively smaller fraction as gene length increases (Figure 4 Supplement 2A) \textbf{(D)} Overlap between major Repeat classes and ATAC peaks for all peak sequence and for inserted sequences (as defined in Figure 5).