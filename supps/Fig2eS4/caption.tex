\textbf{Separability of cell population clusters}.
\textbf{(A)} Definition of separability. Cartoon represents two different single cell clusters as distributions of points. The separability is the ratio of the distance between the centroids to the sum of the "diameter" of each cluster. The diameter of a cluster is calculated as the mean distance to the centroid of the cluster $+$ 3 times the  standard deviation of the distances of each point in the cluster. With this definition, two clusters are "touching" when separability $=$1, overlapping when $<$1, and separate when $>$1. The multi-dimensional distance is computed as 1- Pearson's corr.coef. Note that averaging is expected to improve separability by roughly the square root of the number of cells averaged, hence most of the improved separability in the NeuroSeq data likely reflects averaging. 
\textbf{(B)} Separabilities between cell population clusters for three datasets shown with two different dynamic ranges (color scale; 0-1 for upper row and 0-10 for lower row).  The order of cell population clusters are the same as in Figure 2.