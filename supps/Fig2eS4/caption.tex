\textbf{Separability of cell type clusters}
\textbf{(A)} Definition of separability. Cartoon represents two different single cell clusters as distributions of points. The separability is the ratio of the distance between the centroids to the sum of the "diameter" of each cluster. Here, we calculate the diameter of a cluster using the distances from the centroid of the cluster as the mean distance $+$ 3 times the standard deviation of the distribution of the distances. With this definition, two clusters are "touching" when separability $=$1, overlapping when $<$1, and separate when $>$1. The multi-dimensional distance is computed as 1- Pearson's corr.coef. Note that averaging is expected to improve separability by roughly the square root of the number of cells averaged, hence most of the improved separability in the NeuroSeq data likely reflects a simple effect of averaging. 
\textbf{(B)} Separabilities between cell type clusters for three datasets shown with two different dynamic ranges (color scale; 0-1 for upper row and 0-10 for lower row).  The order of cell type clusters are the same as in Figure 2.