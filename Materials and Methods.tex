\section{Methods and Materials}

\subsection{Mouse Lines}
Mouse lines profiled in this study are summarized in Supplemental Table 1. Most were obtained from GENSAT (Gong 2007) or from the Brandeis Enhancer Trap Collection (Shima 2016). For Cre-driver lines, the Ai3, Ai9 or Ai14 reporter (Madisen 2009) was crossed and offspring hemizygous for Cre and the reporter gene were used for profiling. All experiments were conducted in accordance with the requirements of the Institutional Animal Care and Use Committee at Janelia Research Campus.

\subsection{Atlas}
Animals were anesthetized and perfused with 4\% paraformaldehyde and brains were sectioned at 50 $\mu m$. Every fourth section was mounted on slides and imaged with a slide scanner equipped with 20x objective lens (3DHISTECH; Budapest, Hungary). In house programs were used to adjust contrast and remove shading caused by uneven lighting. Images were converted to a zoomify compatible format for web delivery and are available at neuroseq.janelia.org.

\subsection{Cell Sorting}
Manual cell sorting was performed as described (Hempel 2007, Sugino 2014). Briefly, animals were sacrificed following isoflurane anesthetia, and 300 $\mu m$ slices were digested with pronase E (1mg/ml, P5147; Sigma-Aldrich) for 1 hour at room temperature, in artificial cerebrospinal fluid (ACSF) containing 6,7-dinitroquinoxaline-2,3-dione (20 $\mu M$; Sigma-Aldrich), D-(–)-2-amino-5-phosphonovaleric acid (50 $\mu M$; Sigma-Aldrich), and tetrodotoxin (0.1 $\mu M$; Alomone Labs). Desired brain regions were micro-dissected and triturated with Pasteur pipettes of decreasing tip size. Dissociated cell suspensions were diluted 5-20 fold with filtered ACSF containing fetal bovine serum (1\%; HyClone) and poured over Petri dishes coated with Sylgard (Dow Corning). For dim cells, Petri dishes with glass bottoms were used. Fluorescent cells were aspirated into a micropipette (tip diameter 30-50 $\mu m$ under a fluorescent stereomicroscope (M165FC; Leica), and were washed 3 times by transferring to clean dishes. After the final wash, pure samples were aspirated in a small volume (1~3 $\mu l$) and lysed in 47 $\mu l$ XB lysis buffer (Picopure Kit, KIT0204; ThermoFisher) in 200 $\mu l$ PCR tube (Axygen), incubated for 30 min at 40  $^{\circ}$ on a thermal cycler and then stored at -80$^{\circ}C$. Detailed information on profiled cell groups are provided in Supplemental Table 3.

\subsection{RNA-seq}
Total RNA was extracted (Picopure KIT0204; ThermoFisher). Either 1$\mu l$ of $10^{-5}$ dilution of ERCC spike-in control (\#4456740; Life Technologies) or (number of sorted cells/50) * (1$\mu l$ of $10^{-5}$ dilution of ERCC) was added to the purified RNA and speed-vacuum concentrated down to 5 $\mu l$ and immediately processed for reverse transcription using the NuGEN Ovation RNA-Seq System V2 (\#7102; NuGEN) which yielded 4-8 $\mu g$ of amplified DNA. Amplified DNA was fragmented (Covaris E220) to an average of ~200bp and ligated to Illumina sequencing adaptors with the Encore Rapid Kit (0314; NuGEN).Libraries were quantified with the KAPA Library Quant Kit (KAPA Biosystems) and sequenced on an Illumina HiSeq 2500 with 4 to 32-fold multiplexing (single end, usually 100bp read length, see Supplemental Table 3).

\subsection{RNA-seq analysis}
Adaptor sequences (AGATCGGAAGAGCACACGTCTGAACTCCAGTCAC for Illumina sequencing and CTTTGTGTTTGA for NuGEN SPIA) were removed from de-multiplexed FASTQ data using cutadapt v1.7.1 (http://dx.doi.org/10.14806/ej.17.1.200) with parameters “--overlap=7 --minimum-length=30”. Abundant sequences (ribosomal RNA, mitochondrial, Illumina phiX and low complexity sequences) were detected using bowtie2 (Langmead 2012) v2.1.0 with default parameters. The remaining reads were mapped to UCSC mm10 genome using STAR (Dobin 2012) v2.4.0i with parameters “--chimSegmentMin 15 --outFilterMismatchNmax 3”. 
Reads uniquely mapped to exons were quantified using featureCounts (Subread v...).
For 3' exon quantification, we used bedtools (Quinlan 2010) to extract the exonic sequence up to 1000bp upstream of each TES.
DI and SC were calculated as described in Fig. 3. q-values and log fold changes were calculated in limma following voom transform (Law 2014).
ANOVA was calculated ...
For single-cell RNA-seq comparisons, we downloaded count data and metadata from Tasic et al. (GS...) and Zeisel et al. (...), and decomposed the isocortex cell types profiled in our bulk RNA-seq data set using the single-cell cluster profiles from these studies.
NNLS regression was performed on quantile normalized TPMs using a log-weighted error model (?)... Genes were selected based on intersection top ... ANOVA genes for all 3 studies.
Random forest classification was performed ...
Analysis scripts are available at ...

\subsection{ATAC-seq}
Cells were lysed and nuclei isolated for Tn5 digestion. ATAC libraries for Illumina next-generation sequencing were prepped in accordance with a published protocol (Buenrostro et al., 2013). Library quality was assessed prior to sequencing using bioanalyzer estimates of fragment size distributions as well as qPCR to determine relative enrichment at several common ATAC peaks.

\subsection{ATAC-seq analysis}
Nextera adaptors (CTGTCTCTTATACACATCT) were trimmed from both ends of de-multiplexed FASTQ files using cutadapt with parameters “-n 3 -q 30,30 -m 36 ”. Reads were mapped to UCSC mm10 genome using bowtie2 with parameters “-X2000 --no-mixed --no-discordant”. PCR duplicates were removed using Picard tools (v2.8.1) and reads mapping to mitochondrial DNA, scaffolds, and alternate loci were discarded. BigWig genomic coverage files were generated using bedtools and scaled by the total number of reads per million. For reproducible peaks, liberal peaks were called using HOMER (v4.8.3) with parameters "-style factor -region -size 90 -fragLength 90 -minDist 50 -tbp 0 -L 2 -localSize 5000 -fdr 0.5" and filtered using the Irreproducibility Discovery Rate (IDR) in homer-idr (github.com/karmel/homer-idr.git) with parameters $“--threshold 0.05 --pooled_threshold 0.0125”$. Peak counts and peak patterns were quantified using bedtools.

\subsection{Tissue data}
In addition to cell type-specific data obtained in this study, we analyzed publicly available RNASeq and DNase-seq data using tissue samples. Information on these samples are described in Supplemental Table 2.

\subsection{Annotations}
For reference annotations we used Gencode.vM13 (Harrow 2012) downloaded from http://www.gencodegenes.org/, NCBI RefSeq (Pruitt 2013) and UCSC known genes both downloaded from the UCSC genome browser.

\subsection{Anatomical Region Abbreviations}
Region abbreviations: AOBmi, Accessory olfactory bulb, mitral layer; MOBgl, Main olfactory bulb, glomerular layer; PIR, Piriform area; COAp, Cortical amygdalar area, posterior part; AOBgr, Accessory olfactory bulb, granular layer; MOBgr, Main olfactory bulb, granular layer; MOBmi, Main olfactory bulb, mitral layer; VISp, Primary visual area; AI, Agranular insular area; MOp5, Primary motor area, layer5; VISp6a, Primary visual area, layer 6a; SSp, Primary somatosensory area; SSs, Supplemental somatosensory area; ECT, Ectorhinal area; ORBm, Orbital area, medial part; RSPv, Retrosplenial area, ventral part; ACB, Nucleus accumbens; OT, Olfactory tubercle; CEAm, Central amygdalar nucleus, medial part; CEAl, Central amygdalar nucleus, lateral part; islm, Major island of Calleja; isl, Islands of Calleja; CP, Caudoputamen; CA3, Hippocampus field CA3; DG, Hippocampus dentate gyrus; CA1, Hippocampus field CA1; CA1sp, Hippocampus field CA1, pyramidal layer; SUBd-sp, Subiculum, dorsal part, pyramidal layer; IG, Induseum griseum; CA, Hippocampus Ammon’s horn; PVT, Paraventricular nucleus of the thalamus; CL, Central lateral nucleus of the thalamus; AMd, Anteromedial nucleus, dorsal part; LGd, Dorsal part of the lateral geniculate complex; PCN, Paracentral nucleus; AV, Anteroventral nucleus of thalamus; VPM, Ventral posteromedial nucleus of the thalamus; AD, Anterodorsal nucleus; RT, Reticular nucleus of the thalamus; MM, Medial mammillary nucleus; PVH, Paraventricular hypothalamic nucleus; PVHp, Paraventricular hypothalamic nucleus, parvicellular division; SO, Supraoptic nucleus; DMHp, Dorsomedial nucleus of the hypothalamus, posterior part; ARH, Arcuate hypothalamic nucleus; PVHd, Paraventricular hypothalamic nucleus, descending division; SCH, Suprachiasmatic nucleus; LHA, Lateral hypothalamic area; SFO, Subfornical organ; VTA, Ventral tegmental area; SNc, Substantia nigra, compact part; SCm, Superior colliculus, motor related; IC, Ingerior colliculus; DR, Dorsal nucleus raphe; PAG, Periaqueductal gray; PBl, Parabrachial nucleus, lateral division; PG, Pontine gray; LC, Locus ceruleus; CSm, Superior central nucleus raphe, medial part; AP, Area postrema; NTS, Nucleus of the solitary tract; MV, Medial vestibular nucleus; NTSge, Nucleus of the solitary tract, gelatinous part; DCO, Dorsal cochlear nucleus; NTSm, Nucleus of the solitary tract, medial part; IO, Inferior olivary complex; VII, Facial motor nucleus; DMX, Dorsal motor nucleus of the vagus nerve; RPA, Nucleus raphe pallidus; PRP, Nucleus prepositus; CUL4,5mo, Cerebellum lobules IV-V, molecular layer; CUL4,5pu, Cerebellum lobules IV-V, Purkinje layer; PYRpu, Cerebellum Pyramus (VIII), Purkinje layer; CUL4,5gr, Cerebellum lobules IV-V, granular layer; MOE, main olfactory epithelium; VNO, vemoronasal organ.


