\subsection{Metrics to quantify diversity}
Analysis of expression differences between individual groups is the basis of most profiling efforts. Variance-based metrics, such as Analysis of Variance (ANOVA) F-Value, or coefficient of variation (CV) are commonly used for this purpose. However, these metrics are jointly affected by the pattern of differential expression and the robustness of the differences, and so cannot readily separate these two features (Figure 3,4; Figure 3 Supplement 1). Since these features may differ in their biological significance, we searched for the simplest way to quantitatively separate them, leading us to adopt two easily calculated variants of widely used metrics for differential expression and fold-change that better separate the pattern, or information content, and the robustness of expression differences.

To quantify the contribution of each gene to patterns of cell type diversity, we measured the fraction of cell population pairs in which the gene is differentially expressed. (For differential analysis, the voom-limma framework was used, see Methods). This differentially expressed fraction (DEF) is closely related to the Gini-Simpson diversity index \citep{Simpson_1949} widely used in ecology to measure species diversity in a community (see Figure 3 Supplement 2). DEF ranges from 0 to 1. The maximum observed value of 0.65 indicates that the gene distinguishes 65\% of the pairs, while a value of 0 indicates that the gene distinguishes none (i.e., it is expressed at similar levels in all cells). DEF is easy to calculate and approximates the mutual information (MI) between expression levels and cell types (Figure 3 Supplement 2). 

The robustness of an expression difference depends on its magnitude relative to the underlying noise. Robustness is often quantified as a Signal-to-Noise-Ratio (SNR). Since the signals we are interested in are the gene expression differences distinguishing cell types, we computed the ratio of the mean fold-change expression differences between distinguished pairs to the mean fold-change between undistinguished pairs. This fold-change ratio (FCR) indicates the robustness of pair distinctions, but is independent of the number of pairs distinguished. High FCR genes robustly distinguish cell populations and are therefore suitable as ”marker genes”. %Example marker genes for each profiled cell type are included in Supplementary table/figure XX.

Unlike DEF and FCR, variance-based methods like ANOVA F-values and CV are either affected by both MI and SNR (ANOVA; Figure 4A-C and Figure 3 Supplement 1) or by neither (CV; Figure 3 Supplement 1). The fact that ANOVA does not distinguish between information content and SNR can be appreciated from the fact that high-ANOVA genes (Figure 4A-C) include both high DEF and high FCR genes. Therefore, DEF and FCR are useful because they provide independent measures of the robustness and magnitude of differential expression between cell types.

To determine the types of genes most differentially expressed (highest DEF) and most robustly different (highest FCR) between cell populations, we performed over-representation analysis using the HUGO Gene Groups (\citealt{Braschi_2018}, Figure 4D,E). The most robust expression differences (highest FCR) were those of homeobox transcription factors (TFs; Figure 4D). As might be expected, high DEF genes are enriched for neuronal effector genes including receptors, ion channels and cell adhesion molecules (Figure 4E). Similar results were obtained using the PANTHER gene families \citep{Mi_2016} and Gene Ontology annotations\cite{Ashburner_2000} (Figure 4 Supplement 1).  

Thus, using these two simple metrics we identify synaptic and signaling genes as the most differentially expressed, and homeobox TFs as the most robustly distinguishing families of genes. These two categories of genes drive neuronal diversity by endowing neuronal cell types with specialized signaling and connectivity phenotypes, and by orchestrating cell type-specific patterns of transcription. In addition, their distinct contributions to distinguishing neuronal types suggests possible differences in the regulation of these two categories of genes. 


