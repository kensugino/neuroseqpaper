\subsection{Extracting a compact TF code}
The transcriptional diversity of neuronal cell types arises developmentally through the spatially and temporally staged actions of specific TFs. After development is complete, the identity of diverse cell types must be maintained by the continued expression of key TFs, although these need not be related to the TFs that establish cell fate. In order to better understand how individual TFs encode information about profiled cell types, we extracted the most compact number of TFs needed to specify each of the profiled cell types (or in some cases, a set of highly similar cell types). We reasoned that the constraint of efficiency implicit in extracting a compact code would focus attention on the most important TFs for establishing or maintaining neuronal identity. We identified the TF exhibiting clear “On” an “Off” expression states, whose pattern of expression maximally distinguished cell types in our sample (highest bDI). We then split cell types into two groups according to whether expression of the TF was “On” or “Off.” We repeated this process iteratively on each of the resulting groups, in each case identifying the TF with the maximum DI and bisecting the TFs accordingly. In order to maximize the relevance of the extracted TFs and corresponding cell groupings, at each stage we extracted the TF best able to distinguish the two groups, relative to its ability to distinguish subgroups. The resulting extracted hierarchy of TFs corresponds to a decision tree () for efficiently classifying cell types.

The TFs selected as nodes were frequently genes previously implicated in the development or maintenance of the distinguished cell types. For example, Foxg1, which split forebrain from other cell types, is known to be critically required for normal development of the telencephalon (Xuan et al. 1995; Danesin and Houart, 2012) and is known to function cell autonomously within the olfactory placode for the production of olfactory sensory neurons and all other cells in the olfactory lineage (Duggan et al. 2008). Note that the TF decision tree is based on adult expression, so the retinal ganglion cell (RGC) subtypes sampled are included in the Foxg1- group, even though Foxg1 is known to be expressed in the early eyecup and to contribute to RGC pathfinding (Schulte et al. 2005).

Most, but not all, nodal TFs are well known as key transcriptional regulators (KTRs) of the corresponding cell types in which they are expressed. For example, although level 2 and 3 genes Tbr1(), Satb2(), Egr3(), Isl1() and Emx2(), as well as most of the factors in levels 4 and 5, are each known as KTRs involved in the development and/or maintenance of cell types in corresponding portions of the decision tree, Tox2 has received little prior study in the CNS, although it has recently been identified and replicated as locus of heritability for Major Depressive Disorder (Zeng et al. 2016; PMID:28153336). Hence the list of nodal genes represents a rich source of novel hypotheses about transcriptional regulation in genetically identified cell types. 
The arrangement of cell types along the decision tree reflects known major developmental relationships, but also deviates from those relationships, especially at lower levels of the hierarchy. Presumably this reflects the fact that the same TF is reused as a KTR in multiple lineages, and if those lineages are close enough in terms of other TFs selected they may wind up closer on the tree than warranted by their developmental relationship. We attempted to strengthen the relevance of the tree to the known relationships between cell types by constraining bisections to keep groups of cell types sharing the same anatomical subregion and neurotransmitter phenotype together. We found that we could easily construct such trees, but could not simultaneously enforce the constraint of perfectly dividing groups by nodal TF expression, instead allowing groups that contained a majority with nodal TF expression “On” or “Off” as appropriate. This may reflect cell types in which the expression of the nodal TF is not maintained, or is turned on later. An example tree with the subregion constraint enforced but the nodal expression constraint relaxed is shown as a Supplementary Figure (). Many of the same TFs are identified in both trees. We also examined the effect of relaxing the constraint…Again many of the same genes were selected, although as expected the tree deviated further from known functional relationships. Of the 129 nodal genes identified (78 shown in Fig. 5 and 51 additional genes with matching or perfectly complimentary expression patterns), 106 (82%) and 79 (61%) were identical in the trees with enhanced and no anatomical constraints respectively. The fact that 3/5 of the genes are identical even when no anatomical constraints are included suggests that most of the information used to select nodal genes is already present in expression patterns and does not require additional assumptions about functional relationships between cell types.
