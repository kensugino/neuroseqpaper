\subsection{Genes are elongated by repeat insertions and repeats carry regulatory information}

The above results indicate that gene length is an important contributor to gene expression diversity across cell types. Gene lengths differ widely across species (Figure 7A and Figure 7 Supplement1A), suggesting genes are elongated during evolution. In fact, evolutionary older genes are longer (Figure 7 Supplement 1B; \cite{Grishkevich_2014}). To see how genes can be elongated, we looked into segments of genome different between closely related species (Figure 7B). We compared human and chimp using gorilla as a surrogate common ancestor and extracted segments inserted into human. (Segments exist only in human but not in chimp nor gorilla are called insertion in human, and segments exist in human and gorilla but not in chimp are called deletion in chimp.) Plotted in Figure 7B (left) is the histogram of the length of the inserted segments. Two clear peaks are recognizable. These turned out to be Alu and L1 repeats. Moreover, around 92\% of the base pairs of the inserted segments overlap with repeats (Figure 7A insert; repeats annotation is from repeatmasker.org \cite{Hubley_2015}). Similar results are observed in mouse/rat comparison using guinea pig as a surrogate ancestor (Figure 7B right). These indicate that genes are elongated by mobile element insertions. 

Since long genes have more regulatory elements as indicated by more ATAC-peaks, we asked whether these regulatory elements can originate from mobile elements. As shown in Figure 7C, 56\% of the ATAC peaks overlap with repeats and 75\% of ATAC peaks intersecting with newly inserted segments overlap with repeats, indicating mobile elements/repeats may carry regulatory functions. To explore this possibility, that the mobile elements/repeats form a global regulatory sub-network of gene expression, we fitted gene expression levels with "repeat scores" which are counts of individual repeat within genebody and surrounding 10kbp regions (Figure 7D).  The $R^2$ values for each cell type calculated using test genes (20\%) not used for fitting (Figure 7E, blue) are on average 0.27 and highly larger than shuffled cases (Figure 7E, green/red/orange). Green in Figure 7E corresponds to shuffling of rows of score matrix which yielded no explanation power. Orange and red corresponding to shuffling of the locations of either genes or repeats, which alter relationship between genes and repeats but keeps length information, yielded about 1/3 of the $R^2$ of the original fitting. We found gene length is highly correlated with expression (Figure 7 Suplement 1C;c=0.418:mean Pearson's r between log gene length and expression rank) and SINEs are highly correlated with both gene length (c=0.841) and expression (Figure 7 Supplement 1C; mean c=0.454). Hence, location shuffling likely captured expression variance explainable by gene length, which may be mediated through SINE (and other length dependent repeats). We also fitted gene expression by repeat scores calculated from only upstream or downstream of the genes and found that $R^2$ maxes around 10 to 50 kbp (Figure 7 Supplement 1D), indicating that regulatory information is available outside of the genebody and is concentrated in up to 10\sim 50kbp. Accordingly, $R^2$ for fitting with repeat scores calculated from genebody only or genebody$+/-$100kbp were worse than genebody$+/-$10bp (Figure7 Supplement 1E). 

In summary, genes are elongated by insertions of mobile elements which can carry regulatory functions, can alter expression pattern of the genes and can also increase the capacity to be differentially expressed. 


