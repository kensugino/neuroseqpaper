\subsection{Gene length correlates with nervous system complexity}

The above results indicate that gene length is an important contributor to gene expression diversity, which is a key mechanism underlying cellular diversity. In particular we know that distal regulatory elements (such as those examined within introns above) contribute most to cell type specificity in expression ().  Cellular diversity is a hallmark of the nervous system and is also thought to be critical for the complexity of function. Thus gene length and the mechanisms that drive lengthening over time are potential contributors to the evolution of nervous system complexity. This predicts that species with longer genes will have a larger, more diverse set of neuronal cell types, permitting construction of more complex neural circuits. Consistent with this, we find that the distribution of gene lengths for the best annotated species is broadly correlated with nervous system size and complexity (Figure 7A). This is also true of neuronal genes, (defined in the mouse), which also increase in length with increased nervous system complexity (Figure 7A). An example gene illustrating this relationship (the voltage-gated potasium channel Kcnma1 (slo)) is shown in Figure 7B. 

-acknowledge known expansion of vertebrate introns, known importance of transposons in this process. Motivate looking at which transposons contribute most to elongation of long neuronal genes in mammals.

Then how different species come to have such a different distribution of gene length, as exemplified by the calcium-gated potassium channel Kcnma1 (slo) shown in Fig.7B? To answer this question, we looked into the difference of the genomic sequences of a pair of closest species, human and chimp. We used gorilla as the common ancestor and extracted sequences inserted into human since gorilla. When the histogram of the lengths of the inserted intervals are plotted, two peaks were apparent (Fig.7C). It turned out the shorter peak corresponds to Alu insertions and longer peak corresponds to L1 insertions. Overall, 92% of the base pair of the inserted segments overlap with the RepeatMasker [REF] annotated mobile elements. This and analyses from other groups [cite Yanai] indicate that genes are elongated by insertion of the mobile elements during evolution. Since different species have different sets of mobile elements, it is probable that specific set of mobile elements contributed more in elongation of genes.

[The following is not a result but a hypothesis so may need to put into the discussion]

These observations lead to a hypothesis that the elongation of long genes by insertions of mobile elements in germ cells is an efficient way of creating cell type-, and hence, behavioral- diversity which should increase the adaptability of the population. Specific set of mobile elements in human might have been beneficial in accelerated acquisition of complex brain.
