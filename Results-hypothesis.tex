\subsection{Gene length correlates with nervous system complexity}

The above results indicate that gene length is an important contributor to neuronal diversity and that this correlates with increased numbers of candidate regulatory elements. This also suggests that species with longer genes may have a larger, more diverse set of neuronal cell types, permitting construction of more complex neural circuits. Consistent with this, the distribution of mean gene lengths for the best annotated species is broadly correlated with nervous system size and complexity (Figure 7A). This is especially true of neuronal genes exemplified by targets of REST, which show an even steeper increase in length with increased nervous system complexity (Figure 7A). An example gene illustrating this relationship (the voltage-gated potasium channel Kcnma1 (slo)) is shown in Figure 7B. 

-------------------------------------

Results from the previous section indicate that the gene length is important in creating neuronal diversity. Because the number of cis-elements increases with length, it also suggests that species with longer genes may have more diverse set of neurons, which may form a basis of more complex nervous system. To see whether this is the case, we calculated the distribution of the gene length for well annotated species (Fig.7A). When ordered by upper quartile of the gene lengths (or by medians), it matches to the perceived order of behavioral complexity (i.e. nervous system complexity) of the species. [Fig.7A CHANGE LATIN NAME TO COMMON NAME]

Then how different species come to have such a different distribution of gene length, as exemplified by the voltage-gated pottasium channel Kcnma1 (slo) shown in Fig.7B? To answer this question, we looked into the difference of the genomic sequences of a pair of closest species, human and chimp. We used gorilla as the common ancestor and extracted sequences inserted into human since gorilla. When the histogram of the lengths of the inserted intervals are plotted, two peaks were apparent (Fig.7C). It turned out the shorter peak corresponds to Alu insertions and longer peak corresponds to L1 insertions. Overall, 92% of the base pair of the inserted segments overlap with the RepeatMasker [REF] annotated mobile elements. This and analyses from other groups [cite Yanai] indicate that genes are elongated by insertion of the mobile elements during evolution. Since different species have different sets of mobile elements, it is probable that specific set of mobile elements contributed more in elongation of genes.

[The following is not a result but a hypothesis so may need to put into the discussion]

These observations lead to a hypothesis that the elongation of long genes by insertions of mobile elements in germ cells is an efficient way of creating cell type-, and hence, behavioral- diversity which should increase the adaptability of the population. Specific set of mobile elements in human might have been beneficial in accelerated acquisition of complex brain.
