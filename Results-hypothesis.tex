\subsection{Gene length correlates with nervous system complexity}

The above results indicate that gene length is an important contributor to neuronal diversity and that this is correlated with an increased number of candidate regulatory elements. This also suggests that species with longer genes may have a larger, more diverse set of neuronal cell types, which may in turn permit construction of more complex neural circuits. To see whether this is the case, we calculated the distribution of the gene length for well annotated species (Fig.7A). When ordered by upper quartile of the gene lengths (or by medians), it matches to the perceived order of behavioral complexity (i.e. nervous system complexity) of the species. [Fig.7A CHANGE LATIN NAME TO COMMON NAME]

Then how different species come to have such a different distribution of gene length, as exemplified by the voltage-gated pottasium channel Kcnma1 (slo) shown in Fig.7B? To answer this question, we looked into the difference of the genomic sequences of a pair of closest species, human and chimp. We used gorilla as the common ancestor and extracted sequences inserted into human since gorilla. When the histogram of the lengths of the inserted intervals are plotted, two peaks were apparent (Fig.7C). It turned out the shorter peak corresponds to Alu insertions and longer peak corresponds to L1 insertions. Overall, 92\% of bp of \ inserted segments overlap with the RepeatMasker [REF] annotated mobile elements. This and analyses from other groups [cite Yanai] indicate that genes are elongated by insertion of the mobile elements during evolution. Since different species have different sets of mobile elements, it is probable that specific set of mobile elements contributed more in elongation of genes.
