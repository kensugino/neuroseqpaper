\subsection{Gene length correlates with nervous system complexity}

The above results indicate that gene length is an important contributor to neuronal diversity and that this correlates with increased numbers of candidate regulatory elements. This also suggests that species with longer genes may have a larger, more diverse set of neuronal cell types, permitting construction of more complex neural circuits. Consistent with this, the distribution of mean gene lengths for the best annotated species is broadly correlated with nervous system size and complexity (Figure 7A). An example gene exemplifying this relationship (the voltage-gated potasium channel Kcnma1 (slo)) is shown in Figure 7B. 

Introns are known to elongate through retrotransposition. 
