% Abstract < 150 words 
\begin{abstract} 
The nervous system controls perception, cognition, and behavior through circuits comprised of diverse neuron types. To reveal the diversity of neuronal transcriptomes, we profiled 174 genetically labeled neuronal and 15 non-neuronal cell-types in mouse using RNASeq. To quantify expression diversity we introduced two metrics, differentiation index (DI) and signal contrast (SC). DI quantifies the amount of differential expression and information content. SC quantifies how robust the differential expression is. Genes with high SC's are enriched in homeobox transcription factors. As a family, it contains the highest information on cell-types than other families, consistent with the idea that they are terminal selectors. Genes with high DI's are enriched in long genes, which are characterized by synaptic genes and not expressed in non-neuronal cells. Thus, the diversity of neuronal transcriptomes is mainly realized by long genes. This may have an implication in the evolution of complex nervous systems. 
% currently 140 words 

% remove "useful to other" part? => REMOVED (2017-05-06 KS)
% elaborate on region difference? => No space (2017-05-06 KS)
% homeobox TFs: improve the 'higher information' statement
% elaborate on details on long gene differential expression? => No space (2017-05-06 KS)
% improve last sentence: something like: and identify homeobox genes as master regulators of neuronal diversity => cannot say definitely without transformation experiments (2017-05-06 KS)

\end{abstract}
