\subsection{Homeobox TFs as a family contain the highest information on cell types}

High SC can arise through uniformly high expression levels in ON cell types, uniformly low expression levels in OFF cell types, or both. The average expression levels of homeobox genes (~30 RPKM) are significantly lower than, for example, those of  neuropeptides (~90 RPKM), implying that homeobox TFs have uniformly low expression in OFF cell types. This can, in fact, be readily observed (Figure 4A). We quantified this "OFF noise" (defined as a standard deviation of samples with $FPKM<1$) for all genes (Figure 4B) and found that homeobox genes are enriched among genes with low OFF noise having at least moderate ON expression levels (red dashed region in Figure 4B).

Since tight control of expression may reflect closed chromatin, we measured chromatin accessibility using ATAC-seq (see method). As expected, compared to noisy genes (Figure 4C bottom), genes with low OFF noise were more likely to be without peaks (Figure 4C top, Figure 4D), consistent with the idea that their expression is controlled at the chromatin level.

Functionally, the tight control of homeobox TF expression levels may reflect their known importance as determinants of cell identity, and the fact that establishing and maintaining robust differences between cell types may require tight ON/OFF regulation rather than graded regulation. If they are in fact important in cell type determination, their expression pattern should contain high information on cell types. Although, homeobox TF family was not included in the enriched families in top 1000 DI, as a family it has average DI of 0.21 and 3rd highest following ligand-gated ion channel (0.25) and ion channel (0.22) within PANTHER families with more than 100 members (16th if including all PANTHER families).  

Since binary ON/OFF expression pattern may be more important in cell type specification than graded expression pattern, we calculated a binary version of DI (bDI). Homeobox TF family is the most enriched PANTHER family in the top 1000 bDI genes (Figure 4 Supplement 1A) and also had the highest average bDI among PANTHER families with more than 100 members. 

To see more detailed contribution from subfamilies of TFs, we used TF families in HUGO (Figure 4 Supplement 1B). As expected, homeobox subfamilies had the highest average bDIs within various TF subfamilies. Within the homeobox family, LIM class homeobox genes, known to be critical for specifying spinal cord cell types [REF Shirasaki], have the highest mean DI (Figure 4 Supplement 1B). 

Furthermore, to see how well homeobox TF genes separates cell types compared to other PANTHER family gene sets, 
we calculated separability between the cell types for each of PANTHER family gene sets (Figure 4E). Homeobox TF family has the highest mean separability (5.9) followed by ligand-gated ion channel (4.6) It also has the highest percentage of pairs (95.9\%) with separability$>$2 (Figure 4 Supplement 2C). (Pseudo) distance metric used here to calculate separability was 1-corr.coef., but similar results can be obtained with Euclidean distance as well (Figure 4 Supplement 1D). 

Thus, homeobox TFs has the highest average bDI and best family to separates cell types. This suggests that as has been demonstrated for HOX genes in the vertebrate spinal cord and brainstem \cite{Dasen_2009,Philippidou_2013}, and motoneurons \cite{Kratsios_2017} and somatosensory neurons \cite{Zheng_2015} in C. elegans, homeobox TFs, likely in concert with other TFs, encode the transcriptional diversity of cell types more broadly in the mammalian nervous system.
