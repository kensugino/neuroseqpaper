\subsection{The homeobox TF family contains the highest information regarding cell types}

SC, like SNR, is a ratio between signal and noise, and so can reflect high expression levels in most ON cell types (high signal), low expression levels in most OFF cell types (low noise), or both. Homeobox genes are not among the most abundantly expressed genes. Their average expression levels (~30 FPKM) are significantly lower than, for example, those of neuropeptides (~90 FPKM). This suggests that the high SC of homeobox TFs depend more on low noise than on their high signal. In fact, most  homeobox TFs have uniformly low expression in OFF cell types (Figure 4A). We quantified this "OFF noise" for all genes and found that homeobox genes are enriched among genes that have both low OFF noise and at least moderate ON expression levels (red dashed region in Figure 4B).

Since tight control of expression may reflect closed chromatin, we measured chromatin accessibility using ATAC-seq (see methods). As expected, compared to high-noise genes (Figure 4C bottom), genes with low OFF noise were more likely to have fewer, smaller peaks within their transcription start site (TSS) and gene body (Figure 4C top, Figure 4D), consistent with the idea that their expression is controlled at the level of chromatin accessibility.

Functionally, the tight control of homeobox TF expression levels may reflect their known importance as determinants of cell identity, and the fact that establishing and maintaining robust differences between cell types may require tight ON/OFF regulation rather than graded regulation. If they are, in fact, important drivers of cell type-specific differences, their expression pattern should be highly informative about cell types. However, the homeobox family was not identified on the basis of a particularly high DI (Figure 3E; mean 0.21; rank 16 among X) compared to, for example, cyclic nucleotide-gated ion channels (mean DI=0.31, highest) or GABA receptors (0.29, 2nd). We infer that this is due to the fact that graded expression differences also contribute to DI. Since binary ON/OFF expression patterns may be more critical for cell type specification than graded expression patterns, we calculated a binary version of DI (bDI; see methods). With this metric, the homeobox TF family is the most enriched PANTHER family among the top 1000 bDI genes (Figure 4 Supplement 1A) and had the 2nd highest average bDI (0.07) among PANTHER families after neuropeptides (0.08).

To more explicitly measure the contributions from subfamilies of TFs, we computed bDI across TF families and subfamilies in HUGO (Figure 4 Supplement 1B). As expected, homeobox subfamilies occupied most of the top positions. The LIM domain subfamily of homeobox genes had the highest mean bDI, consistent with its known role in specifying spinal cord and brainstem cell types [REF Shirasaki]. 

[Needs better intro to distance/separation; comes out of nowhere]. Furthermore, to see how well homeobox TF genes separates cell types compared to other PANTHER family genes, we looked at distances between the cell types calculated using each of PANTHER family (Figure 4E). Homeobox TF family has the highest mean between cell type distance and highest z-score compared to random sampling. [ refer to decorrelation? (supp C) ]

Thus, homeobox TFs have the highest average bDI and provide the best separation of profiled cell types. A subset of homeobox TFs, the HOX genes, are known to specify cell types in the vertebrate spinal cord and brainstem \cite{Dasen_2009,Philippidou_2013}, as well as motoneurons \cite{Kratsios_2017} and sensory neurons \cite{Zheng_2015} in C. elegans. Our analysis reveals that the larger family of homeobox TFs, play a broader role in transcriptional diversity of cell types across the mammalian nervous system.
