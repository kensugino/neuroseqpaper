\subsection{Gene Categories Contributing to Neuronal Diversity}

(Fig.4B will be moved to the end of the fig., Fig4A will be split into 4A and 4B, the rest will be renamed accordingly.)

To determine the types of genes contributing to neuronal diversity, we identified the PANTHER gene families with the highest DI (most differentially expressed) or highest SC (most robust differences). Since the read count (and hence variability) is affected by gene length [REF Oshlack], we restricted counts to 1kb from the 3' end of the gene (Fig4.Supp.1A). 

As expected, high DI genes are enriched for neuronal effector genes including receptors, ion channels and cell adhesion molecules (Fig.4A). This is consistent with the observations that neurons differ most dramatically in their chemical and electrical signaling properties and in their synaptic connectivity.

The least noisy expression differences (highest SC) were those of homeobox transcription factors (Fig.4B), leading to overrepresentation for the larger categories of TFs and DNA binding proteins. High SC can reflect uniformly high expression levels in ON cell types, uniformaly low expression levels in OFF cell types, or both. The average expression level of homeobox genes--around 30 FPKM--is much lower than, for example that of  neuropeptides (~90 FPKM), implying that homeobox TFs have uniformly low expression in OFF cell types. This can, in fact, be readily observed (Fig.4C). We quantified this "OFF noise" (defined as a standard deviation of samples with $FPKM<1$) for all genes (Fig.4D) and found that homeobox genes are enriched (Fig.4E) among genes with low OFF noise and at least moderate ON expression levels (red dashed region in Fig.4D ).

Since tight control of expression is usually associated with a closed chromatin [REF], we checked ATAC-seq data (see method), for OFF hoemobox genes (Fig.4F). As expected, compared to noisy genes (Fig.4G), they were more likely to be without peaks (Fig.4H [REPLACE WITH A NEW ONE]), consistent with the idea that their expression is controlled at the chromatin level.

Functionally, the tight control of expression levels of homeobox genes relative to other genes is also consistent with the idea that they are especially important determinants of cell identity.  which suggests their expression pattern must contain high information content regarding cell type. Since we did not see homeobox in the top enriched families in high DI genes, we calculated a binarized version of DI (bDI), where a pair of cell types are only distinguished by ON or OFF states (i.e. disregarding difference in graded expression). As expected, in this metric, homeobox family is at the top of the enriched families (Fig.4I [ADD THIS PANEL BACK]). Within the homeobox family, LIM class homeobox genes, which are known to be critical in specifying the spinal cord cell types [REF Shirasaki], have the highest mean bDI (Fig.4 Supp.1B). Furthermore, we calculated mutual information between the cell types and the entire genes of the PANTHER families and found that homeobox family and parent TF families contain much higher information than expected from random (Fig.4J)

Thus, using SC analysis, we objectively identified homeobox family as well as parent TF families as likely determinants of the cell types. [BETTER WAY TO SAY THIS? CONNECTION LOGIC TO THE NEXT SECTION?]
