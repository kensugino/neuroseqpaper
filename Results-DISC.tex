\subsection{Gene Categories Contributing to Neuronal Diversity}

To determine the types of genes contributing to neuronal diversity, we identified the PANTHER gene families (Figure 3E) with the highest DI (most cell types distinguished) or highest SC (most robust differences). Since the read count (and hence variability) is affected by gene length \cite{Oshlack_2009}, we restricted counts to 1kb from the 3' end of the gene (Figure 3 Supplement 2). 

As expected, high DI genes are enriched for neuronal effector genes including receptors, ion channels and cell adhesion molecules (Figure 3E). This is consistent with the observations that neurons differ most dramatically in their chemical and electrical signaling properties and in their synaptic connectivity.

The least noisy expression differences (highest SC) were those of homeobox transcription factors (Figure 3E), leading to overrepresentation for the larger categories of TFs and DNA binding proteins. High SC can arise through uniformly high expression levels in ON cell types, uniformly low expression levels in OFF cell types, or both. The average expression levels of homeobox genes (~30 RPKM) are significantly lower than, for example, those of  neuropeptides (~90 RPKM), implying that homeobox TFs have uniformly low expression in OFF cell types. This can, in fact, be readily observed (Figure 4A). We quantified this "OFF noise" (defined as a standard deviation of samples with $RPKM<1$) for all genes (Figure 4B) and found that homeobox genes are enriched among genes with low OFF noise having at least moderate ON expression levels (red dashed region in Figure 4B).

Since tight control of expression may reflect closed chromatin, we measured chromatin accessibility using ATAC-seq (see method), for OFF homeobox genes (Figure 4C,D). As expected, compared to noisy genes (Figure 4G), they were more likely to be without peaks (Fig.4H [REPLACE WITH A NEW ONE]), consistent with the idea that their expression is controlled at the chromatin level.

Functionally, the tight control of homeobox TF expression levels may reflect their known importance as determinants of cell identity, and the fact that establishing and maintaining robust differences between cell types may require ON/OFF regulation, rather than graded regulation of some key TFs. To quantify the information content present in On/Off expression patterns across cell types, we calculated a binary (On/Off) version of DI both for individual TFs and for families of TFs (Methods). Homeobox TFs were most enriched for individual TFs with high DI [ADD THIS PANEL BACK]) and were even more enriched when calculating DI across families (Figure 4 Supplement 1). This greater degree of enrichment when calculating across families arises because the expression patterns of individual family members are highly orthogonal, so that even though a single homeobox TF may specify only one or a small number of cell types, the ensemble of homeobox TFs encode information about diverse cell types. Within the homeobox family, LIM class homeobox genes, known to be critical for specifying spinal cord cell types [REF Shirasaki], have the highest mean DI (Fig.4 Supp.1B). Furthermore, we calculated mutual information between the cell types and the entire gene sets of the PANTHER families and found that homeobox family and parent TF families contain much higher information than expected from random (Fig.4J). This suggests that as has been demonstrated for HOX genes in the vertebrate spinal cord and brainstem \cite{Dasen_2009,Philippidou_2013}, and motoneurons \cite{Kratsios_2017} and somatosensory neurons \cite{Zheng_2015} in C. elegans, homeobox TFs, likely in concert with other TFs, encode the transcriptional diversity of cell types more broadly in the mammalian nervous system.
